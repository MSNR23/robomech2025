\section{緒言}
不整地踏破バリカンロボットや階段などの段差が多い建造物内において二足歩行は真価を発揮する.
物を搬送する機械はこれまでにも数多く開発され,それらの多くは
車輪で移動する.しかし,車輪は階段などの段差の移動が難しく運用上に制限がある.
同じく車輪で移動する搬送用機械として電動車椅子\cite{refer5},\cite{refer6}が挙げられる.
電動車椅子も同様に,段差の移動が難しく運用上に制限がある.
これを脚なら,踏み出し動作により段差によらず移動が可能となり,
運用上の制限を格段に少なくできる.
そのため,我々は人や物を搬送できる二足歩行ロボットの開発を行っている\cite{refer9}.\\
 人を搬送する二足歩行ロボットの先行研究として,WLシリーズ\cite{refer1},
搭乗型脚車輪ロボット\cite{refer2},HUBO\cite{refer3},i-foot\cite{refer7}が挙げられる.
これらは人を載せた状態での移動を実現している.
また,物を搬送する二足歩行ロボットとしてAgility
Robotics製のdigit\cite{refer4}が注目されている.
ここで,物や人を搬送する際に着目すべき点として「積載物を倒さず運ぶ」
ことが挙げられる.倒さず運ぶためには積載物が転がらず,滑り出さない
ことが重要である.これは人を載せた際の乗り心地にも影響する.
二足歩行は前後・左右方向に揺れるため積載物を転がさず滑らせないで
運ぶことは難しい.
しかし,先行研究において積載物の転がりと滑りを抑制する歩行方法は
検討されていない.\\
 本論文では二足歩行ロボットに物や人を搭載し,
搬送する場合を想定して,積載物が転がらず滑らない条件を定式化する.
加えて,積載物が転がらず滑らない歩行条件を定式化する.そして,
検討した歩行方法をシミュレータにて
\figref{robot_load.eps}に示すモデルを用いて検証する.
\begin{figure}[t]
    \begin{center}
    \includegraphics[width=70mm]{./fig/robot_load.eps}
    % \vspace*{-0.5cm}%鉛直方向のFig.の位置
    \caption{A load-carrying bipedal robot}
    % \vspace*{-1cm}
    \figlabel{robot_load.eps}
  \end{center}
\end{figure}
% 脚なら段差によらずに移動が可能となり
% 物を搬送するロボットはこれまでにも数多く研究され,それらの多くは
% 車輪で移動する.(参考文献6個)
% 物を搬送するロボットの移動方法として車輪が挙げられる.
% 車輪で移動する

% しかし,先行研究において積載物の転がりと滑りの抑制は考慮されておらず,
% 手やベルト等で固定することが前提となっている.\\

% 物の搬送において物を落とさず様々な場所へ運ぶことは重要である.

% そのため,物を搬送するロボットとして二足歩行ロボットが注目されている(digit).

% そのため,・階段昇降が可能な電動車いすの欠点を列挙する
% 森林地帯などの狭く溝や高低差が多い不整地や階段などの段差が多い建造物内に
% おいて二足歩行は真価を発揮する.
% 物を搬送するロボットは主に車輪によって移動する.
% 車輪だと階段などの段差を登れず,移動できる範囲が狭い.
% 同様の状況下で人を搬送する乗り物として,電動車いすが挙げられる.
% 電動車いすも車輪での移動となるため,移動できる範囲が狭い.
% 脚は踏破性が高く,階段などの段差も登ることができる.
% したがって,我々は以上の問題を解決するために人や物を搬送できる
% 二足歩行ロボットの開発を行っている.\\
%  搭乗型二脚モビリティの先行研究としてはWLシリーズ\cite{refer1},
% 搭乗型脚車輪ロボット\cite{refer2},HUBO\cite{refer3}が挙げられ,
% これらは人を載せた状態での移動を実現している.
% 物や人を搬送する際に着目すべき点として「物や人を落とさず運ぶ」
% ことが挙げられる.落とさず運ぶためには積載物が転がらず,滑り出さない
% ことが前提として挙げられる.歩行は前後,左右方向に揺れるため
% 積載物を転がらず,滑り出さないで運ぶことは難しい.
% 先行研究において,積載物の転がりと滑りの抑制は考慮されていない.
% したがって,本論文では二足歩行ロボットに物や人を搭載し,
% 搬送する場合を想定して,積載物が転がらず滑らない条件を定式化する.
% そして,積載物が転がらず滑らない歩行方法を検討する.そして,
% 検討した歩行方法をシミュレータにて検証する.

% ・搬送用ロボットと電動車いすに着目している
% ・搬送用ロボットの先行研究を列挙
% ・問題点を列挙
% ・問題点を解決する(なぜ二足歩行なのか)
% ・搭乗型ロボットの先行研究を列挙
% ・問題点を列挙
% ・問題点を解決する(なぜ転がりと滑りを防止したいのか)
% ・線形倒立振子モードを使う理由も
% ・本論文では二足歩行ロボットに物や人を搭載し,搬送する場合を想定して,
% 積載物が転がらず滑らない条件を定式化する.そして,積載物が転がらず滑らない
% 歩行方法を線形倒立振子モード(LIPM)(梶田先生論文)\cite{refer1}にて検討する.そして,
% 検討した歩行方法をシミュレータにて検証する.

