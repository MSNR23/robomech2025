\documentclass[dvipdfmx]{jarticle}
\usepackage{robomech}
\usepackage{graphicx}
\usepackage{ikuo}%%便利コマンド集.
\usepackage{siunitx}
\usepackage{jtygm}
\usepackage{bm}
\usepackage{balance}
\usepackage{amssymb}
\usepackage{here}
\usepackage{amsmath}
\usepackage{otf}%難しい「高」を追加するためのusepackage
\usepackage{url}
\newcommand{\FIGDIR}{./fig}%図を置くディレクトリを指定する.

\begin{document}
\makeatletter
\title{投擲物の重さや身体のパラメータに応じた投擲フォーム戦略の研究}
{\vspace{-7 mm} }
{Research on Throwing Form Strategies\\according to Weight of the Thrown Objects or Body Parameters}
{}

\author{
\begin{tabular}{ll}
 \hspace{1zw}学\hspace{1zw}伊藤浩平 (東京農工大)& 
 \hspace{1zw}\hspace{1zw}森下克幸 (東京農工大)\\ 
 \hspace{1zw}○正\hspace{1zw}水内郁夫 (東京農工大)\\
 % ※協賛・後援団体の会員資格で発表される場合は「正・学」は不要です.
 \end{tabular}
 % &\\
 \vspace{1zh} \\
 \begin{tabular}{l}
{\small Kohei ITO, TUAT, k-ito-rm25@mizuuchi.lab.tuat.ac.jp
}\\
{\small Katsuyuki MORISHITA, TUAT, morishita@mizuuchi.lab.tuat.ac.jp
}\\
{\small Ikuo MIZUUCHI, TUAT, ikuo@mizuuchi.lab.tuat.ac.jp}
\end{tabular}
}
\makeatother

\abstract{ \small 
There are many sports in which throwing movements are performed, but the throwing form differs between competitions and individuals. Although research specific to a particular sport has been conducted, a general theory related to the strategies on which various throwing forms are based has not been established. In this research, we derive various throwing forms using reinforcement learning, and we examine, consider, and discuss their strategies. Verification has identified throwing form strategies according to the weight of the thrown objects and arm length. The method used in this research was able to also show to be useful in discussing throwing form strategies.
}
 
\date{} % 日付を出力しない
\keywords{Throwing form, Strategy, Reinforcement learning}

\maketitle
\thispagestyle{empty}
\pagestyle{empty}

\small

% \section{緒言}
不整地踏破バリカンロボットや階段などの段差が多い建造物内において二足歩行は真価を発揮する.
物を搬送する機械はこれまでにも数多く開発され,それらの多くは
車輪で移動する.しかし,車輪は階段などの段差の移動が難しく運用上に制限がある.
同じく車輪で移動する搬送用機械として電動車椅子\cite{refer5},\cite{refer6}が挙げられる.
電動車椅子も同様に,段差の移動が難しく運用上に制限がある.
これを脚なら,踏み出し動作により段差によらず移動が可能となり,
運用上の制限を格段に少なくできる.
そのため,我々は人や物を搬送できる二足歩行ロボットの開発を行っている\cite{refer9}.\\
 人を搬送する二足歩行ロボットの先行研究として,WLシリーズ\cite{refer1},
搭乗型脚車輪ロボット\cite{refer2},HUBO\cite{refer3},i-foot\cite{refer7}が挙げられる.
これらは人を載せた状態での移動を実現している.
また,物を搬送する二足歩行ロボットとしてAgility
Robotics製のdigit\cite{refer4}が注目されている.
ここで,物や人を搬送する際に着目すべき点として「積載物を倒さず運ぶ」
ことが挙げられる.倒さず運ぶためには積載物が転がらず,滑り出さない
ことが重要である.これは人を載せた際の乗り心地にも影響する.
二足歩行は前後・左右方向に揺れるため積載物を転がさず滑らせないで
運ぶことは難しい.
しかし,先行研究において積載物の転がりと滑りを抑制する歩行方法は
検討されていない.\\
 本論文では二足歩行ロボットに物や人を搭載し,
搬送する場合を想定して,積載物が転がらず滑らない条件を定式化する.
加えて,積載物が転がらず滑らない歩行条件を定式化する.そして,
検討した歩行方法をシミュレータにて
\figref{robot_load.eps}に示すモデルを用いて検証する.
\begin{figure}[t]
    \begin{center}
    \includegraphics[width=70mm]{./fig/robot_load.eps}
    % \vspace*{-0.5cm}%鉛直方向のFig.の位置
    \caption{A load-carrying bipedal robot}
    % \vspace*{-1cm}
    \figlabel{robot_load.eps}
  \end{center}
\end{figure}
% 脚なら段差によらずに移動が可能となり
% 物を搬送するロボットはこれまでにも数多く研究され,それらの多くは
% 車輪で移動する.(参考文献6個)
% 物を搬送するロボットの移動方法として車輪が挙げられる.
% 車輪で移動する

% しかし,先行研究において積載物の転がりと滑りの抑制は考慮されておらず,
% 手やベルト等で固定することが前提となっている.\\

% 物の搬送において物を落とさず様々な場所へ運ぶことは重要である.

% そのため,物を搬送するロボットとして二足歩行ロボットが注目されている(digit).

% そのため,・階段昇降が可能な電動車いすの欠点を列挙する
% 森林地帯などの狭く溝や高低差が多い不整地や階段などの段差が多い建造物内に
% おいて二足歩行は真価を発揮する.
% 物を搬送するロボットは主に車輪によって移動する.
% 車輪だと階段などの段差を登れず,移動できる範囲が狭い.
% 同様の状況下で人を搬送する乗り物として,電動車いすが挙げられる.
% 電動車いすも車輪での移動となるため,移動できる範囲が狭い.
% 脚は踏破性が高く,階段などの段差も登ることができる.
% したがって,我々は以上の問題を解決するために人や物を搬送できる
% 二足歩行ロボットの開発を行っている.\\
%  搭乗型二脚モビリティの先行研究としてはWLシリーズ\cite{refer1},
% 搭乗型脚車輪ロボット\cite{refer2},HUBO\cite{refer3}が挙げられ,
% これらは人を載せた状態での移動を実現している.
% 物や人を搬送する際に着目すべき点として「物や人を落とさず運ぶ」
% ことが挙げられる.落とさず運ぶためには積載物が転がらず,滑り出さない
% ことが前提として挙げられる.歩行は前後,左右方向に揺れるため
% 積載物を転がらず,滑り出さないで運ぶことは難しい.
% 先行研究において,積載物の転がりと滑りの抑制は考慮されていない.
% したがって,本論文では二足歩行ロボットに物や人を搭載し,
% 搬送する場合を想定して,積載物が転がらず滑らない条件を定式化する.
% そして,積載物が転がらず滑らない歩行方法を検討する.そして,
% 検討した歩行方法をシミュレータにて検証する.

% ・搬送用ロボットと電動車いすに着目している
% ・搬送用ロボットの先行研究を列挙
% ・問題点を列挙
% ・問題点を解決する(なぜ二足歩行なのか)
% ・搭乗型ロボットの先行研究を列挙
% ・問題点を列挙
% ・問題点を解決する(なぜ転がりと滑りを防止したいのか)
% ・線形倒立振子モードを使う理由も
% ・本論文では二足歩行ロボットに物や人を搭載し,搬送する場合を想定して,
% 積載物が転がらず滑らない条件を定式化する.そして,積載物が転がらず滑らない
% 歩行方法を線形倒立振子モード(LIPM)(梶田先生論文)\cite{refer1}にて検討する.そして,
% 検討した歩行方法をシミュレータにて検証する.


% \section{積載物が転がらず滑らない条件の定式化}
\subsection{積載物とロボットのモデル化}
\label {2.1}
\begin{figure}[t]
    \begin{center}
    \includegraphics[width=70mm]{./fig/robot_load_model_2.eps}
    % \vspace*{-0.5cm}%鉛直方向のFig.の位置
    \caption{Rigid model of a robot body and load}
    % \vspace*{-1cm}
    \figlabel{robot_load_model.eps}
  \end{center}
  \end{figure}
  
ロボットに搭載した物が転がり始めず,滑り始めない条件を定式化するために
積載物とロボットを\figref{robot_load_model.eps}に示す剛体にモデル化した.
積載物モデルの底面形状は長方形とし,質量を${m}_{\mathrm{l}}$
,重心位置を$\bm{c}_{\mathrm{l}}$とする.
一方,ロボットモデルは積載物との接触部分のみを考慮するために,
ボディのみを長方形の剛体としてモデル化し,重心位置を
$\bm{c}_{\mathrm{r}}$とする.
また,ロボットボディの並進運動による影響のみに着目するために,
ロボットボディの慣性モーメントは無限大と
みなし,回転運動は生じないものとする.
積載物とロボットボディ間に作用する
摩擦力の合力を$\bm{f}$,垂直抗力の合力を$\bm{N}$で表し,それら反力の
作用点(圧力中心)を$\bm{p}$とする.$\bm{g}$は重力
加速度である.\\%[m/${\mathrm{s}^2}$]
 ここで,積載物が転がり始めるのは摩擦力$\bm{f}$よる重心周りのモーメント
と垂直抗力$\bm{N}$による重心周りのモーメントの和が$\bm{0}$より大きくなった
ときである.
したがって,積載物が転がり始めない摩擦力の範囲は\equref{1.1}で表される.
\begin{eqnarray}
  \equlabel{1.1}
  \bm{f} &\leq&-{[\bm{s}_{\mathrm{l}}]}_{\mathrm{x}}^{-1}{[\bm{s}_{\mathrm{l}}]}_{\mathrm{x}}\bm{N}\\
  \bm{s}_{\mathrm{l}} &=&\bm{p}-\bm{c}_{\mathrm{l}}\\
  \bm{s}_{\mathrm{l}} &=&\left[
    \begin{array}{ccc}
      {s}_{\mathrm{lx}} & {s}_{\mathrm{ly}} & {s}_{\mathrm{lz}}
    \end{array}
    \right]^{\mathrm{T}}
\end{eqnarray}
ここで,${[ ]}_{\mathrm{x}}$は歪対称行列を表す演算子,
$\bm{s}_{\mathrm{l}}$は積載物の重心から見た時の圧力中心の位置である.

積載物が転がり始める瞬間において,圧力中心$\bm{p}$は積載物底面の辺と端に
移動し,辺と端を回転中心として転がり始める.そのため,転がり始める瞬間
の${s}_{\mathrm{lx}}$,${s}_{\mathrm{ly}}$は積載物の重心投影点と辺までの長さに,
${s}_{\mathrm{lz}}$は重心の底面からの高さに等しくなる.\\

\subsection{積載物の転がりと滑りの条件分の定式化}
転がりが先に起こる場合と滑りが先に起こる場合の条件分を定式化する.
\equref{1.1}で表される積載物が転がり始める
摩擦力の絶対値よりも最大静止摩擦力の絶対値が大きい場合は転がりが
先に起こり,小さい場合は滑りが先に起こる.
したがって,\equref{1.1}より転がりが先に起こる場合と,
滑りが先に起こる場合の条件分は以下で表される.\\
\begin{eqnarray}
  \equlabel{1.2}
  |-{[\bm{s}_{\mathrm{l}}]}_{\mathrm{x}}^{-1}{[\bm{s}_{\mathrm{l}}]}_{\mathrm{x}}\bm{e}_{\mathrm{z}}|\leq{\mu}
\end{eqnarray}
ここで,$\mu$は積載物とロボットボディ間の最大静止摩擦係数,
$\bm{e}_{\mathrm{z}}$はz成分のみ1となる単位ベクトルである.
右辺が$\mu$以下であれば転がりの方が先に起こり,滑りは生じない.
$\mu$より大きければ滑りの方が先に起こり,転がりは生じない.\\
 ここで,$x$方向のみの場合を考えると
\equref{1.2}は\equref{1.3}で表される.\\
\begin{eqnarray}
  \equlabel{1.3}
  \frac{{s}_{\mathrm{lx}}}{{s}_{\mathrm{lz}}}\leq{\mu}
\end{eqnarray}
\equref{1.3}より積載物の転がりと滑りのどちらが先に起こるかは,
積載物をロボットに乗せる前に判断できる.
% 転がりが先に起こる場合は,\equref{1.1}で表される積載物が転がり始める
% 摩擦力の絶対値よりも最大静止摩擦力の絶対値が大きい場合である.
% 一方で滑りが先に起こる場合は,
% 転がり始める摩擦力の絶対値よりも最大静止摩擦力の絶対値が小さい場合である.

\subsection{積載物が転がり始めない条件の定式化}
\label {2.3}
積載物の転がり始めない条件について考える.
積載物とロボットボディ間に生じる摩擦力は主に積載物に作用する重力とロボットボディ
の重心加速度が起因で生じている.また,転がりが生じる瞬間においては積載物とロボットボディ
の間に滑りは生じていないため,${\ddot{\bm{c}}_{\mathrm{l}}}={\ddot{\bm{c}}_{\mathrm{r}}}$が成立する.
したがって,転がりが生じる瞬間におけるロボットボディの加速度の範囲は\equref{1.1}より
\equref{1.4}で表される.\\
\begin{eqnarray}
  \equlabel{1.4}
  {\ddot{\bm{c}}_{\mathrm{rH}}} &\leq& -{[\bm{s}_{\mathrm{l}}]}_{\mathrm{x}}^{-1}{[\bm{s}_{\mathrm{l}}]}_{\mathrm{x}}({\ddot{\bm{c}}_{\mathrm{rV}}}+\bm{g})\\
  \bm{c}_{\mathrm{r}} &=&\left[
    \begin{array}{ccc}
      {c}_{\mathrm{rx}} & {c}_{\mathrm{ry}} & {c}_{\mathrm{rz}}
    \end{array}
    \right]^{\mathrm{T}}\\
  \bm{c}_{\mathrm{rH}} &=&\left[
    \begin{array}{ccc}
      {c}_{\mathrm{rx}} & {c}_{\mathrm{ry}} & 0
    \end{array}
    \right]^{\mathrm{T}}\\
  \bm{c}_{\mathrm{rV}} &=&\left[
    \begin{array}{ccc}
      0 & 0 & {c}_{\mathrm{rz}}
    \end{array}
    \right]^{\mathrm{T}}\\
  \bm{g} &=&\left[
    \begin{array}{ccc}
      0 & 0 & g
    \end{array}
    \right]^{\mathrm{T}}
\end{eqnarray}
% \begin{bmatrix}
%   1 & 0 & 0 \\
%   0 & 1 & 0\\
%   0 & 0 & 0  
% \end{bmatrix}
% \bm{c}_{\mathrm{r}}  \\
% \bm{c}_{\mathrm{rV}} &=&
% \begin{bmatrix}
%   0 & 0 & 0 \\
%   0 & 0 & 0\\
%   0 & 0 & 1  
% \end{bmatrix}
% \bm{c}_{\mathrm{r}} 
 ここで,$x$方向のみの場合を考えると
\equref{1.4}は\equref{1.5}で表される.\\
\begin{eqnarray}
  \equlabel{1.5}
  {\ddot{c}_{\mathrm{rx}}} &\leq& \frac{{s}_{\mathrm{lx}}}{{s}_{\mathrm{lz}}}({\ddot{c}_{\mathrm{rz}}}+{g})
\end{eqnarray}
したがって,\equref{1.5}の範囲内になるようにロボットボディの加速度を制御すれば,
積載物の転がりを抑制できる.

\subsection{積載物が滑り始めない条件の定式化}
\label {2.4}
積載物の滑り始めない条件について考える.
転がりと同様に,滑りが生じる瞬間においては
${\ddot{\bm{c}}_{\mathrm{l}}}={\ddot{\bm{c}}_{\mathrm{r}}}$が成立する.
したがって,滑りが生じる瞬間におけるロボットボディの加速度は\equref{1.6}で表される.\\
\begin{eqnarray}
  \equlabel{1.6}
  |{\ddot{\bm{c}}_{\mathrm{rH}}}| &\leq& \mu({\ddot{c}_{\mathrm{rz}}}+g)
\end{eqnarray}
 ここで,$x$方向のみの場合を考えると
\equref{1.6}は\equref{1.7}で表される.\\
\begin{eqnarray}
  \equlabel{1.7}
  {\ddot{c}_{\mathrm{rx}}} &\leq& \mu({\ddot{c}_{\mathrm{rz}}}+{g})
\end{eqnarray}
したがって,\equref{1.7}の範囲内になるようにロボットボディの加速度を制御すれば,
積載物の滑りを抑制できる.
%  したがって,転がりと同様に積載物が転がり始めないためには\equref{1.7}を
% 満たすようにロボットボディの加速度を制御すればいい.

% 転がりよりも滑りよりが先に起こる場合について考える.\\
%  滑りが生じる瞬間における積載物の加速度は\equref{1.1},〇〇より
% \equref{1.6}で表される.\\
%  このとき,積載物とロボットボディの間に滑りは生じているいないため,〇〇を満たす.
% したがって,転がりが生じる瞬間におけるロボットの加速度は\equref{1.6}で表される.\\
%  ここで,わかりやすくするためにロボットの加速度\equref{1.6}を二次元の場合で考える.
% このとき\equref{1.6}は\equref{1.7}で表される.\\
%  したがって,積載物が転がり始めないためには\equref{1.7}を満たすように
% ロボットのボディの加速度を制御すればいい.


% \section{LIPMにおける転がりと滑りの抑制方法の検討}
\label {3}
\begin{figure}[t]
  \begin{center}
  \includegraphics[width=70mm]{./fig/LIPM_2.eps}
  % \vspace*{-0.5cm}%鉛直方向のFig.の位置
  \caption{Linear inverted pendulum mode}
  % \vspace*{-1cm}
  \figlabel{LIPM.eps}
\end{center}
\end{figure}

積載物が転がり始めず,滑り始めないロボットの歩行方法を
検討するために,ロボットを\figref{LIPM.eps}に示す線形倒立振子モード
(LIPM)\cite{refer8}として考える.${p}_{\mathrm{sup}}$は
前後方向における足の接地位置,${c}_{\mathrm{rx}}$は前後方向
重心位置,${c}_{\mathrm{rz}}$は重心高さである.
また,\ref{2.1}節にてモデル化したロボットボディの重心位置とLIPMにおける
ロボットの重心位置は足の慣性を無視し,等しいと考える.\\
 LIPMではロボットの重心高さを一定に保つため,
物を上下に揺らさずに搬送できる.加えて,\equref{1.5}と\equref{1.7}に
含まれる上下方向重心加速度$\ddot{c}_{\mathrm{rz}}$は0 m/${\mathrm{s}^2}$となり,
式が簡単になる.\\
 支持脚切替時の前後方向重心位置を${c}_{\mathrm{rz}}(0)$,
前後方向歩幅を${w}_{\mathrm{x}}=2({c}_{\mathrm{rz}}(0)-{p}_{\mathrm{sup}})$
とすると,LIPMにおけるロボットの重心加速度は支持脚切替時に最大値
${\ddot{c}_{\mathrm{rx}}}(0)=\frac{{w}_{\mathrm{x}}}{2{c}_{\mathrm{rz}}}g$を取る.

\subsection{積載物が転がり始めない歩行方法の検討}
\label {3.1}
まずは積載物が転がり始めない歩行方法をLIPMにおける重心加速度から検討する.
LIPMにおけるロボットの重心加速度の最大値より,歩幅${w}_{\mathrm{x}}$と重心高さ
${c}_{\mathrm{rz}}$によってロボットの重心加速度の最大値が調節できることがわかる.
したがって,\equref{1.5}とロボットの重心加速度の最大値より
積載物が転がり始めない歩幅と重心高さの範囲はそれぞれ
\equref{1.8},\equref{1.9}で表される.\\
\begin{eqnarray}
  \equlabel{1.8}
  {w}_{\mathrm{x}} &\leq& 2\frac{{s}_{\mathrm{lx}}}{{s}_{\mathrm{lz}}}{c}_{\mathrm{rx}}\\
  \equlabel{1.9}
  {c}_{\mathrm{rz}} &\geq& \frac{{s}_{\mathrm{lz}}}{2{s}_{\mathrm{lx}}}{w}_{\mathrm{x}}
\end{eqnarray}
以上から,歩幅,重心高さを調節することで積載物の転がりは抑制できる.
%  また,初期重心位置と接地位置の差分は歩幅に相当するため,\equref{1.8}は
% 歩幅を用いて\equref{1.9}で表すこともできる.\\
%  また,接地位置は足に面積を持つ場合や両足接地時はZMPに相当する.
\subsection{積載物が滑り始めない歩行方法の検討}
\label {3.2}
積載物が滑り始めない歩行方法をLIPMにおける重心加速度から検討する.
転がりと同様に,\equref{1.7}とロボットの重心加速度の最大値より
積載物が滑り始めない歩幅と重心高さの範囲はそれぞれ
\equref{2.0},\equref{2.1}で表される.\\
\begin{eqnarray}
  \equlabel{2.0}
  {w}_{\mathrm{x}} &\leq& 2\mu{c}_{\mathrm{rx}}\\
  \equlabel{2.1}
  {c}_{\mathrm{rz}} &\geq& \frac{{w}_{\mathrm{x}}}{2\mu}
\end{eqnarray}
以上から,転がりと同様に歩幅,重心高さを調節することで積載物の滑りは抑制できる.
%  また,初期重心位置と接地位置の差分は歩幅に相当するため,\equref{2.0}は
% 歩幅を用いて\equref{2.2}で表すこともできる.\\
%  また,接地位置は足に面積を持つ場合や両足接地時はZMPに相当する.
% 以上から,転がりと同様にZMP,歩幅,重心高さを調節することで積載物の滑りは抑制できる.
 
% \section{ロボット歩行時における転がりと滑りの抑制方法の検証}
\subsection{シミュレーション条件}
\begin{figure}[t]
    \begin{center}
    \includegraphics[width=60mm]{./fig/robot_3.eps}
    % \vspace*{-0.5cm}%鉛直方向のFig.の位置
    \caption{Model of bipedal robot}
    % \vspace*{-1cm}
    \figlabel{robot.eps}
  \end{center}
\end{figure}

\begin{figure}[t]
    \begin{minipage}[b]{0.45\linewidth}
      \centering
      \includegraphics[width=1.0\linewidth]{./fig/loadA.eps}%
      % \hspace{-10truemm}%水平方向のFig.の位置
      \vspace*{-0.1cm}%鉛直方向のFig.の位置
      \caption{Model of load A}
      \figlabel{loadA.eps}
    \end{minipage}
    \begin{minipage}[b]{0.45\linewidth}
      \centering
      \includegraphics[width=1.0\linewidth]{./fig/loadB.eps}%
    %  \hspace{10truemm}%水平方向のFig.の位置 
      \caption{Model of load B}
      \figlabel{loadB.eps}
    \end{minipage}
    %\caption{subcaptionを用いて図を並べる}
  \end{figure}

  前章で求めた歩幅と重心高さによって積載物が
  転がり始めず,滑り始めないで歩行が可能かを確認するために
シミュレーション上にて検証を行った.使用したシミュレータは
Choreonoidである.使用したロボットのモデルを\figref{robot.eps}に示す.
ロボットモデルは股ヨー・ロール・ピッチ軸,膝ピッチ軸,足首ロール・ピッチ軸の
片足6自由度,全12自由度で構成される.
% ロボットモデルの大きさは人を載せられる程度に設定した.
ロボットボディの慣性モーメントはLIPMにできるだけ近づけるために
$10^4$ kg${\mathrm{m}^2}$に設定した.
積載物のモデルは2つ用意し,\equref{1.3}を用いてパラメータを設定した.
それぞれ,\figref{loadA.eps}に示す転がりが先に起こる場合のモデルをA,\figref{loadB.eps}に示す
滑りが先に起こる場合ののモデルをBとした.
ロボットモデルと2つの積載物モデルの詳細を\tabref{table1}に示す.\\
 歩幅の検証の際には,重心高さを0.7 mに設定した.よって,
\tabref{table1}と\equref{1.8},\equref{2.0}より
転がり始めない歩幅と滑り始めない歩幅は同じく0.35 m以下になった.\\
 重心高さの検証の際には,歩幅を0.3 mに設定した.よって,
\tabref{table1}と\equref{1.9},\equref{2.1}より
転がり始めない重心高さと滑り始めない重心高さは同じく0.68 m以上になった.\\
 以上より検証条件を\tabref{table2}に示すように設定した.
積載物の縦幅は$2{s}_{\mathrm{lz}}$,横幅は$2{s}_{\mathrm{lx}}$に相当する.
\tabref{table2}を満たすように5 s間ロボットを積載物を載せた状態で歩行させる.
% モデル誤差や横方向加速度の影響等を考慮した上で,\tabref{table2}に示すように設定した.
\subsection{ロボット歩行時の積載物の転がり抑制の検証}
\subsubsection{歩幅を調節した結果}
\label {4.1.1}
歩幅を調節することで積載物の転がりを抑制できるか確認を行った.
歩幅0.3 m,0.4 mで歩行した際の積載物Aのピッチ角を\figref{pitch_0.3_0.4_r.pdf}に示す.\\
 \figref{pitch_0.3_0.4_r.pdf}より,歩幅0.35 m以上の歩幅0.4 mで歩行した際は
積載物のピッチ角の最大値が7.4 degとなり,転がりが生じた.
一方で,歩幅0.35 m以下の歩幅0.3 mで歩行した際は積載物のピッチ角の最大値が0.16 degとなり,
歩幅0.4 mと比べてピッチ角が97.8 \%抑えられた.
ピッチ角0.16degはボディへのめり込みによるものであり,
転がりによるものではないと考えられる.
したがって,\equref{1.8}により求めた歩幅の範囲内において転がりを防止できた.\\
%  また,\figref{xdisp_0.3_0.4_r.pdf}より歩幅0.3 mの時は相対変位の最大値が
% 約5.9 mmとなっており,滑りも抑制されている.
% 一方で歩幅0.4 mの時は相対変位の最大値が約30.9 mmとなっている.
% これは,ピッチ角の転がりが生じたことで積載物の重心位置が変化した結果である.
%  歩幅0.3 m,0.4 mで歩行した際の積載物Aのピッチ角とロボットボディと
% 積載物間の相対変位をそれぞれ\figref{pitch_0.3_0.4_r.pdf}と
% \figref{xdisp_0.3_0.4_r.pdf}に示す.\\
\subsubsection{重心高さを調節した結果}
\label {4.1.2}
重心高さを調節することで積載物の転がりを抑制できるか確認を行った.
重心高さ0.7 m,0.55 mで歩行した際の積載物Aのピッチ角を
\figref{pitch_0.7_0.55_r.pdf}に示す.\\
 \figref{pitch_0.7_0.55_r.pdf}より,
重心高さ0.68 m以下の歩幅0.55 mで歩行した際は
積載物のピッチ角の最大値が90 degとなり,転がりが生じた後に落下した.
一方で,重心高さ0.68 m以上の重心高さ0.7 mで歩行した際は積載物のピッチ角の
最大値が0.16 degとなり,歩幅0.4 mと比べてピッチ角が99.8 \%抑えられた.
\ref{4.1.1}節と同様に,ピッチ角0.16degはボディへのめり込みによるものであり,
転がりによるものではないと考えられる.
したがって,\equref{1.9}により求めた重心高さの範囲内において転がりを防止できた.\\
%  また,\figref{xdisp_0.7_0.55_r.pdf}より歩幅0.3 mの時は相対変位の最大値が約5.9 mmとなって
% おり,滑りも抑制されている.一方で歩幅0.4 mの時は相対変位の最大値が約804 mmとなっている.
% これは,積載物が落下したことによって生じたものである.
%  重心高さ0.7 m,0.55 mで歩行した際の積載物Aのピッチ角とロボットボディと
% 積載物間の相対変位をそれぞれ\figref{pitch_0.7_0.55_r.pdf}と\figref{xdisp_0.7_0.55_r.pdf}
% に示す.\\
\begin{table}[t]
    % \vspace*{0.4cm}%鉛直方向のFig.の位置
    \caption{Parameters of the model}
    \tablabel{table1}
    \begin{center}
        
    \scalebox{0.88}{
    \begin{tabular}{|l | r | r | r|}%1) 2a) 2b) 3) 振幅 振動数 波形 

        \hline
        モデル名&質量&大きさ&最大静止摩擦係数\\ 
        \hline
        ロボット&20 kg&全高 1.2 m&-\\ 
        
        積載物A&0.5 kg&縦・横幅 0.4,0.1 m&0.5\\         
        
        積載物B&0.5 kg&縦・横幅 0.3,0.1 m&0.25\\ 
        \hline
    \end{tabular}
    }
    \end{center}
\end{table}

\begin{table}[t]
    % \vspace*{0.4cm}%鉛直方向のFig.の位置
    \caption{Verification condition}
    \tablabel{table2}
    \begin{center}
        
    \scalebox{0.9}{
    \begin{tabular}{|l | r | r | r | r |}%1) 2a) 2b) 3) 振幅 振動数 波形 

        \hline
        場合分&使用モデル&調整したパラメータ&パラメータ値\\ 
        \hline
        転がり&積載物A&歩幅&0.3 m,0.4 m\\ 
        転がり&積載物A&重心高さ&0.55 m,0.7 m\\
        滑り&積載物B&歩幅&0.3 m,0.4 m\\         
        滑り&積載物B&重心高さ&0.55 m,0.7 m\\ 
        \hline
    \end{tabular}
    }
    \end{center}
\end{table}

\subsection{ロボット歩行時の積載物の滑り抑制の検証}
\subsubsection{歩幅を調節した結果}
\label {4.1.3}
歩幅を調節することで積載物の滑りを抑制できるか確認を行った.
歩幅0.3 m,0.4 mで歩行した際のロボットボディと
積載物B間の相対変位を\figref{xdisp_0.3_0.4_s.pdf}に示す.\\
 \figref{xdisp_0.3_0.4_s.pdf}より,歩幅0.35 m以上の歩幅0.4 mで歩行した際は
ロボットボディと積載物B間の相対変位の最大値が21.2 mmとなった.
一方で,歩幅0.35 m以下の歩幅0.3 mで歩行した際は相対変位の最大値が16.3 mmとなり,
歩幅0.4 mと比べて相対変位が23 \%抑えられた.
したがって,\equref{2.0}により求めた歩幅の範囲内において滑りを軽減できた.
滑りが生じた原因としては,ロボットボディの横方向加速度,ロボットのモデル化誤差,
Choreonoidの滑りのモデルの誤差等が考えられる.\\
%  また,\figref{pitch_0.3_0.4_s.pdf}より歩幅0.3 m,0.4 mの時のピッチ角の最大値が
% それぞれ0.09 deg,0.06 degとなっており,転がりは抑制されている.
%  歩幅0.3 m,0.4 mで歩行した際の積載物Bのピッチ角とロボットボディと
% 積載物間の相対変位をそれぞれ\figref{pitch_0.3_0.4_s.pdf}と\figref{xdisp_0.3_0.4_s.pdf}
% に示す.\\

\subsubsection{重心高さを調節した結果}
重心高さを調節することで積載物の滑りを抑制できるか確認を行った.
重心高さ0.7 m,0.55 mで歩行した際のロボットボディと
積載物B間の相対変位を\figref{xdisp_0.7_0.55_s.pdf}に示す.\\
 \figref{xdisp_0.7_0.55_s.pdf}より,重心高さ0.68 m以下の重心高さ0.55 mで
歩行した際はロボットボディと積載物B間の相対変位の最大値が31.4 mmとなった.
一方で,重心高さ0.68 m以下の重心高さ0.7 mで歩行した際は相対変位の最大値が
16.3 mmとなり,重心高さ0.55 mと比べて相対変位が48 \%抑えられた.
したがって,\equref{2.1}により求めた重心高さの範囲内において滑りを軽減できた.
滑りが生じた原因としては,\ref{4.1.3}節と同様であると考えられる.\\
%  また,\figref{pitch_0.7_0.55_s.pdf}より重心高さ0.7 m,0.55 mの時のピッチ角の最大値が
% それぞれ0.09 deg,0.012 degとなっており,転がりも抑制されている.
%  重心高さ0.7 m,0.55 mで歩行した際の積載物のピッチ角とロボットボディと
% 積載物間の相対変位をそれぞれ\figref{pitch_0.7_0.55_s.pdf}と\figref{xdisp_0.7_0.55_s.pdf}
% に示す.\\

\begin{figure}[tb]
    \begin{center}
    \includegraphics[width=90mm]{./fig/pitch_0.3_0.4_r.pdf}
    % \vspace*{-0.5cm}%鉛直方向のFig.の位置
    \caption{Pitch angle of load when the step length is adjusted}
    % \vspace*{-1cm}
    \figlabel{pitch_0.3_0.4_r.pdf}
\end{center}
\end{figure}
% \begin{figure}[tb]
%     \begin{center}
%     \includegraphics[width=90mm]{./fig/xdisp_0.3_0.4_r.pdf}
%     % \vspace*{-0.5cm}%鉛直方向のFig.の位置
%     \caption{歩幅を調節した時の積載物Aとロボットボディの相対変位}
%     % \vspace*{-1cm}
%     \figlabel{xdisp_0.3_0.4_r.pdf}
% \end{center}
% \end{figure}
\begin{figure}[tb]
    \begin{center}
    \includegraphics[width=90mm]{./fig/pitch_0.7_0.55_r.pdf}
    % \vspace*{-0.5cm}%鉛直方向のFig.の位置
    \caption{Pitch angle of load when the CoG height is adjusted}
    % \vspace*{-1cm}
    \figlabel{pitch_0.7_0.55_r.pdf}
\end{center}
\end{figure}
% \begin{figure}[tb]
%     \begin{center}
%     \includegraphics[width=90mm]{./fig/xdisp_0.7_0.55_r.pdf}
%     % \vspace*{-0.5cm}%鉛直方向のFig.の位置
%     \caption{重心高さを調節した時の積載物Aとロボットボディの相対変位}
%     % \vspace*{-1cm}
%     \figlabel{xdisp_0.7_0.55_r.pdf}
% \end{center}
% \end{figure}
% \begin{figure}[tb]
%     \begin{center}
%     \includegraphics[width=90mm]{./fig/pitch_0.3_0.4_s.pdf}
%     % \vspace*{-0.5cm}%鉛直方向のFig.の位置
%     \caption{歩幅を調節した時の積載物Bのピッチ角}
%     % \vspace*{-1cm}
%     \figlabel{pitch_0.3_0.4_s.pdf}
% \end{center}
% \end{figure}
\begin{figure}[t]
    \begin{center}
    \includegraphics[width=90mm]{./fig/xdisp_0.3_0.4_s.pdf}
    % \vspace*{-0.5cm}%鉛直方向のFig.の位置
    \caption{Displacement between the robot and the load when the step length is adjusted}
    % \vspace*{-1cm}
    \figlabel{xdisp_0.3_0.4_s.pdf}
\end{center}
\end{figure}
% \begin{figure}[tb]
%     \begin{center}
%     \includegraphics[width=90mm]{./fig/pitch_0.7_0.55_s.pdf}
%     % \vspace*{-0.5cm}%鉛直方向のFig.の位置
%     \caption{重心高さを調節した時の積載物Bのピッチ角}
%     % \vspace*{-1cm}
%     \figlabel{pitch_0.7_0.55_s.pdf}
% \end{center}
% \end{figure}
\begin{figure}[tb]
    \begin{center}
    \includegraphics[width=90mm]{./fig/xdisp_0.7_0.55_s.pdf}
    % \vspace*{-0.5cm}%鉛直方向のFig.の位置
    \caption{Displacement between the robot and the load when the CoG height is adjusted}
    % \vspace*{-1cm}
    \figlabel{xdisp_0.7_0.55_s.pdf}
\end{center}
\end{figure}



\section{緒言}
投擲動作を行うスポーツは数多く存在するが,野球と砲丸投げのように競技によって投擲フォームは異なり,さらに同一競技内であっても個人によって投擲フォームは異なる.競技や個人によって投擲フォームが異なる要因として,投擲物や身体といった投擲フォームに関連するパラメータの違いが挙げられる.これまで投擲フォームに関する研究例として,眞田の野球におけるオーバーハンドスローとサイドハンドスローの球速の比較\cite{sanada},黒松らの砲丸投げグライド投法における投擲フォーム改善前後の飛距離の比較\cite{kuromatsu}などがある.また,投擲に関する総合性能の研究例として,蔭山らの野球における体格や背筋力と投球速度の関係\cite{kageyama},坪井らの砲丸投げにおける投射速度・投射角と飛距離の関係\cite{tsuboi}などがある.スポーツにおいて総合性能向上は最も重要な要素の一つである.これらの研究はある一つの競技に特定した研究である.しかし,さまざまな投擲フォームがどのような戦略の基で成立しているのかに関する汎用的な理論は確立されていない.本研究ではシミュレーションにおいてさまざまなパラメータに応じた投擲フォームを導出・比較することで,さまざまな投擲フォームの戦略を検討・考察・議論することを目的とする.\\
\section{強化学習を用いた投擲フォーム導出}
投擲フォームの導出方法について,本研究では強化学習による最適化手法を採用した.投擲フォームは時々刻々と全身の運動連鎖\cite{burkhart}によるダイナミクスが変化するため,明示的な解を求めることは困難である.運動連鎖とは,ある関節の動作が隣接する関節に影響を与え,運動エネルギーを伝達していく運動効果である.強化学習は明示的な解がなく詳細なパラメータ設定が求められる投擲フォーム導出において有効な手段である.
\subsection{強化学習手法}
本研究では,強化学習の手法の1つであるQ学習\cite{watkins}を採用した.Q学習において,Q値は\equref{1}で更新する.
\begin{eqnarray}
  \equlabel{1}
  Q(s,a)=(1-\alpha)Q(s,a)+\alpha(r+\gamma \mathrm{max}Q(s',a'))
\end{eqnarray}
\subsection{行動選択方法}
本研究では,行動選択方法として$\varepsilon$-greedy法\cite{greedy}を採用した.$\varepsilon$-greedy法では,$\varepsilon$(0$\leq$$\varepsilon$$\leq$1)の確率で全ての行動からランダムに行動を選択し,1-$\varepsilon$の確率でルールの価値が最も高い行動を選択する.本研究において$\epsilon$は\equref{2}であり,エピソードが進むにつれてランダム値を選択する確率を小さくする設定とした.
\begin{eqnarray}
  \equlabel{2}
  \epsilon = 0.3 \times 0.99^{(\mathrm{episode} + 1)}
\end{eqnarray}
\figt{5.1.eps}{width=0.5\hsize}{3D Rigid 2 Links(stabilization of the trunk)}
\section{3次元剛体2リンクによるパラメータに応じた投擲フォームの導出・比較と戦略の考察}
\subsection{動力学モデル}
本章で用いた動力学モデルを\figref{5.1.eps}に示す.物理エンジンMuJoCo\cite{mujoco}に標準搭載されているhumanoidモデル「Unitree G1」\cite{unitreeg1}を改変し,人間の腕を肩関節3自由度と肘関節1自由度の計4自由度から構成される3次元剛体2リンクとしてモデル化した.なお,\figref{5.1.eps}において,肩関節から肘関節までを上腕リンク,肘関節から手先までを前腕リンクとし,手首や指の自由度は0とした.また,体幹リンクも自由度0とし,世界座標に固定した.運動方程式はMuJoCoで内部的に解いた.Runge-Kutta法により数値積分し運動学を解くことで,3次元剛体2リンクの角度や角速度を計算した.
\subsection{強化学習の設定}
状態について,状態変数は8つとし,肩関節ピッチ軸周りの角度$\theta_{p}$,角速度$\dot{\theta}_{p}$,肩関節ロール軸周りの角度$\theta_{r}$,角速度$\dot{\theta}_{r}$,肩関節ヨー軸周りの角度$\theta_{y}$,角速度$\dot{\theta}_{y}$,肘関節の角度$\theta_{e}$,角速度$\dot{\theta}_{e}$とした.
角度について,それぞれの可動範囲の角度は,肩関節ピッチ軸周りが$-135$ deg $\leq\theta_{p}\leq45$  deg,肩関節ロール軸周りが$-135$ deg $\leq\theta_{r}\leq-35$ deg,肩関節ヨー軸周りが$-150$ deg $\leq\theta_{y}\leq180$ deg,肘関節が$-20$ deg $\leq\theta_{e}\leq90$ deg とした.しかし,人間の各関節は互いに干渉するため,組み合わせ次第では人間が不可能な姿勢となる.そのため,スプライン補間\cite{spline}を用いて肩関節ピッチ軸周りとロール軸周りの角度によって肩関節ヨー軸周りの可動範囲が変動するように設定した.また,各関節の角速度については,いずれも$-10.0$ m/s $\le$ $\dot{\theta}_{i}$ $\le$ 10.0 m/s($i = p,r,y,e$)とした.
分割数は各角度が5分割,各角速度が2分割であり,全ての状態を$5^{4}\times 2^{4}=10000$通りで表すことができる.\\
\begin{table}[t]
  \begin{center}
    \caption{Parameters which are used for 2D Rigid 2 Links simulation(0.14 kg vs 7.24 kg)}
    \tablabel{1}
    \scalebox{0.9}[0.9]{
    \begin{tabular}{c|c|c|c}
      \hline
      Parameters & Unit & Values of 0.14 kg & Values of 7.24 kg \\
      \hline
      $l_{1}$ & m & 0.32 & 0.32\\
      $l_{2}$ & m & 0.44 & 0.44\\
      $l_{g1}$ & m & 0.16 & 0.16 \\
      $l_{g2}$ & m & 0.21 & 0.40 \\
      $m_{1}$ & kg & 1.96 & 1.96\\
      $m_{2}$ & kg & 1.68 & 8.78 \\
      $I_{1}$ & kg$\cdot$$\mathrm{m}^2$ & 0.017 & 0.017 \\
      $I_{2}$ & kg$\cdot$$\mathrm{m}^2$ & 0.15 & 1.59 \\
      \hline
    \end{tabular}
    }
  \end{center}
\end{table}
\figt{5.9.eps}{width=1.0\hsize}{Reward progress in 3D Rigid 2 Links(left: 0.14 kg thrown object,right: 7.24 kg thrown object)}
行動は,全625通りに設定した.肩の各関節にかかるトルクを正2通り,0,負2通りの計5通り,同様に肘関節にかかるトルクも正2通り,0,負2通りの計5通りとした.
これにより,Qテーブルは$10000 \times 625=6250000$通りで表すことができる.\\
報酬は投擲物の飛距離を報酬とした.投擲物のモデル化は行っていないため,投射中の投擲物に生じる空気抵抗等は考慮しないものとする.\\
飛距離の計算にあたり,ピッチ軸方向,ロール軸方向,ヨー軸方向の3方向成分の手先速度をMuJoCoより取得し,それぞれ$v_{p}$,$v_{r}$,$v_{y}$とした.手先速度3成分の合成$v_{syn}$は,
\begin{eqnarray}
  \equlabel{3}
  v_{syn} = \sqrt{{v_{p}}^{2} + {v_{r}}^{2} + {v_{y}}^{2}}
\end{eqnarray}
また,ピッチ・ロール軸平面の手先速度成分の合成$v_{pr}$は,
\begin{eqnarray}
  \equlabel{4}
  v_{pr} = \sqrt{{v_{p}}^{2} + {v_{r}}^{2}}
\end{eqnarray}
であり,\equref{4}より投射角$\theta_{v}$は,
\begin{eqnarray}
  \equlabel{5}
  \theta_{v} = \arctan2(\frac{v_{y}}{v_{pr}})
\end{eqnarray}
各ステップ時のヨー軸成分の座標$h_{y}$を手先高さとする.しかし,この座標は肩関節を原点とした際の値であり,本来の手先高さは地面から肩関節までの高さを考慮する必要がある.よって,身長を$L$とした際の手先高さ$h$は,
\begin{eqnarray}
  \equlabel{6}
  h = 0.818L + h_{y}
\end{eqnarray}
リリース時の手先高さを考慮した投射時間$t$は,
\begin{eqnarray}
  \equlabel{7}
  t = \frac{v_{syn}\sin\theta_{v} + \sqrt{{v_{syn}}^2\sin^2\theta_{syn} + 2gh}}{g}
\end{eqnarray}
以上より,飛距離$D$は\equref{3},\equref{6},\equref{7}を用いて,
\begin{eqnarray}
  \equlabel{8}
  D = v_{syn} \cdot \cos\theta_{v} \cdot t
\end{eqnarray}
また,罰則として累積消費エネルギー$CE$を採用し,\equref{9}で計算した.累積消費エネルギーの計算にあたり,$\tau_{p}$は肩関節ピッチ軸周りに与えるトルク、$\tau_{r}$は肩関節ロール軸周りに与えるトルク、$\tau_{y}$は肩関節ヨー軸周りに与えるトルク、$\tau_{e}$は肘関節に与えるトルクである。累積消費エネルギーは肩関節3軸周りと肘関節の消費エネルギーの合計とした.
\begin{eqnarray}
  \equlabel{89}
  CE\approx\Sigma|\tau_{i}(t)\cdot\dot{\theta}_{i}|\cdot\Delta t \quad (i=p, r, y, e)
\end{eqnarray}
以上より,報酬の設計は,\equref{10}とした.
\begin{eqnarray}
  \equlabel{10}
  R = D - scale \times CE
\end{eqnarray}
 その他の設定について,\equref{1}における学習率$\alpha$を0.1,割引率$\gamma$を0.9に設定した.1タイムステップは0.001 sであり,1エピソード内のステップ数は3000,すなわち,3 sとした.エピソード数に関しては,10000エピソードの中で最も報酬が高いエピソードを採用した.
\figt{5.11.eps}{width=0.93\hsize}{Throwing form from start to release of 3D Rigid 2 Links from a plane perpendicular to the pitch axis(0.14 kg thrown object,1.72 m tall human)}
\figt{5.13.eps}{width=0.93\hsize}{Throwing form from start to release of 3D Rigid 2 Links from a plane perpendicular to the pitch axis(7.24 kg thrown object,1.72 m tall human)}
\figt{5.7.eps}{width=0.9\hsize}{Transition of shoulder and elbow torque from start to release of 3D Rigid 2 Links(left: 0.14 kg thrown object,right: 7.24 kg thrown object)}
\figt{5.15.eps}{width=0.9\hsize}{Transition of hand height relative shoulder from start to release of 3D Rigid 2 Links(left: 0.14 kg thrown object,right: 7.24 kg thrown object)}
\subsection{投擲物の重さに応じた投擲フォームの結果・考察}
身体のサイズは,身長 1.72 m,体重 70 kgの人間の各部位のサイズとした.投擲物は,野球の硬式球と砲丸の重さを参考に,$0.14$ kg\cite{horiuchi} と$7.24$ kg\cite{haq} とした.\\
 重さ0.14 kg の投擲物と重さ7.24 kg の投擲物での学習の際に用いる各パラメータを\tabref{1}に示す\cite{irving}.\tabref{1}において,$l_{1}$は上腕リンク長さ,$l_{2}$は前腕リンク長さ,$l_{g1}$は上腕リンクの重心までの長さ,$l_{g2}$は前腕リンクの重心までの長さ,$m_{1}$は上腕リンク重さ,$m_{2}$は前腕リンク重さ,$I_{1}$は上腕リンクの重心周りの慣性モーメント,$I_{2}$は前腕リンクの重心周りの慣性モーメントである.重力加速度$g$を9.8 $\mathrm{m/s^{2}}$とする.肩関節3軸周りに与えるトルクについては,$-40$ N$\cdot$m,$-20$ N$\cdot$m,0,20 N$\cdot$m,40 N$\cdot$mから,肘関節に与えるトルクについては$-30$ N$\cdot$m,$-15$ N$\cdot$m,0,15 N$\cdot$m,30 N$\cdot$mから\equref{4}に基づいて選択したが,肩関節ヨー軸周りのトルクに関しては,スプライン補間によって設定した可動範囲に基づき,トルクも補完する設定とした.各関節の粘性係数について,肩関節ピッチ軸周りの粘性係数$b_{p}$は1.0,肩関節ロール軸周りの粘性係数$b_{r}$は0.8,肩関節ヨー軸周りの粘性係数$b_{y}$は0.5,肘関節の粘性係数$b_{e}$は0.2とした.
以上の設定により学習を行い,投擲物の重さに応じた遠投をするための投擲フォームの比較を行った.\\
 初期状態について,$\theta_{p}$,$\theta_{r}$,$\theta_{y}$,$\theta_{e}$はランダム値,$\dot{\theta}_{p}$,$\dot{\theta}_{r}$,$\dot{\theta}_{y}$,$\dot{\theta}_{e}$は0とした.これにより,最も報酬が高くなる際の初期姿勢を導出することができる.投擲物に応じた遠投するための投擲フォーム戦略の結果・考察について述べる.\figref{5.11.eps}はピッチ軸に垂直な面から見た際の重さ0.14 kg の投擲物を遠投するための投擲フォーム,\figref{5.13.eps}は重さ7.24 kg の投擲物を遠投するための投擲フォームである.リリース瞬間は1エピソードの中で最も報酬が高いステップとし,投擲フォームは投擲開始からリリースまでとした.重さ0.14 kg の投擲物を遠投するための投擲フォームのリリース瞬間は0.489 s で,投擲物の飛距離が70.17 m,重さ7.24 kg の投擲物を遠投するための投擲フォームのリリース瞬間は0.447 s で,投擲物の飛距離が6.42 mであった.\\
 \figref{5.7.eps}は各関節のトルクの推移,\figref{5.15.eps}は肩を基準とした際の手先高さの推移である.\figref{5.7.eps},\figref{5.15.eps}において,左図は重さ0.14 kg の投擲物の際で,右図は重さ7.24 kg の投擲物の際の各時系列である.これらより,投擲物の重さに応じた投擲フォームの比較によりみられる戦略の違いは,「腕の押し出し度合い」であると考えられる.\figref{5.15.eps}において,重さ0.14 kg の投擲物の際はリリース前に一度手先が下がってから一気に高くなるが,重さ7.24 kg の投擲物の際はほぼ手先高さに変化がみられない.この波形の直線度が高いほど腕の押し出し度合いが高い.また,0.35 s あたりから,運動連鎖よって肩関節ピッチ軸周りの関節が持つエネルギーをヨー軸周りの関節に伝達している傾向がみられた.よって,重さ0.14 kg の投擲物を遠投するための投擲フォームは,運動連鎖による肩関節の回転を重要視した戦略が考えられる.一方,重さ7.24 kg の投擲物を遠投するための投擲フォームは,運動連鎖による前腕へのエネルギーの伝達により,肘を伸展させるエネルギーを大きくする傾向がみられた.よって,肘の伸展や前腕を重要視した腕の押し出し度合いの高い戦略が考えられる.
\begin{table}[tb]
  \begin{center}
    \caption{Parameters which are used for 2D Rigid 2 Links simulation(1.72 m vs 1.90 m)}
    \tablabel{2}
    \scalebox{0.9}[0.9]{
    \begin{tabular}{c|c|c|c}
      \hline
      Parameters & Unit & Values of 1.72 m & Values of 1.90 m \\
      \hline
      $l_{1}$ & m & 0.32 & 0.35 \\
      $l_{2}$ & m & 0.44 & 0.48 \\
      $l_{g1}$ & m & 0.16 & 0.18 \\
      $l_{g2}$ & m & 0.21 & 0.23 \\
      $m_{1}$ & kg & 1.96 & 1.96\\
      $m_{2}$ & kg & 1.68 & 1.68\\
      $I_{1}$ & kg$\cdot$$\mathrm{m}^2$ & 0.017 & 0.018 \\
      $I_{2}$ & kg$\cdot$$\mathrm{m}^2$ & 0.020 & 0.099 \\
      \hline
    \end{tabular}
    }
  \end{center}
\end{table}
\figt{5.10.eps}{width=1.0\hsize}{Reward progress in 3D Rigid 2 Links(left:1.72 m tall human,right:1.90 m tall human)}
\subsection{腕の長さに応じた投擲フォームの結果・考察}
身体のサイズは,身長1.72 m,体重70 kgの人間と,身長1.90 m,体重70 kg の各部位のサイズとした.投擲物は,野球の硬式球を参考に,0.14 kgとした.\\
 身長1.72 m の人間に基づいた腕の長さでの投擲の際と身長1.90 m の人間に基づいた腕の長さでの学習の際に用いる各パラメータを\tabref{2}に示す.重力加速度$g$を9.8 $\mathrm{m/s^{2}}$とする.各関節に与えるトルクと各関節の粘性係数の設定については,投擲物の重さに応じた投擲フォームの比較の際と同様とした.
以上の設定により学習を行い,腕の長さに応じた遠投をするための投擲フォームの比較を行った.\\
 初期状態について,投擲物の重さによる投擲フォームの比較の際と同様の設定とした.腕の長さに応じた遠投をするための投擲フォーム戦略の結果・考察について述べる.
\figref{5.11.eps}はピッチ軸に垂直な面から見た際の身長1.72 m の人間に基づいた腕の長さによる遠投をするための投擲フォーム,\figref{5.16.eps}は身長1.90 m の人間に基づいた腕の長さによる遠投をするための投擲フォームである.リリース瞬間は1エピソードの中で最も報酬が高いステップとし,投擲フォームは投擲開始からリリースまでとした.身長1.72 m の人間に基づいた腕の長さによる遠投をするための投擲フォームのリリース瞬間は0.489 s で,投擲物の飛距離が70.17 m,身長1.90 m の人間に基づいた腕の長さによる遠投をするための投擲フォームのリリース瞬間は1.109 s で,投擲物の飛距離が92.16 mであった.
\figref{5.8.eps}は各関節のトルクの推移,\figref{5.23.eps}は各関節の角速度の推移である.\figref{5.8.eps},\figref{5.23.eps}において,左図は身長1.72 m の人間に基づいた腕の長さの際の,右図は身長1.90 m の人間に基づいた腕の長さの際の各時系列である.\\
\figt{5.16.eps}{width=0.93\hsize}{Throwing form from start to release of 3D Rigid 2 Links from a plane perpendicular to the pitch axis(0.14 kg thrown object,1.90 m tall human)}
\figt{5.8.eps}{width=0.9\hsize}{Transition of shoulder and elbow torque of 3D Rigid 2 Links from start to release (left:1.72 m tall human,right:1.90 m tall human)}
 比較の結果,身長1.72 m の人間に基づいた腕の長さの 0.28 s からリリースまでと身長1.90 m の人間に基づいた腕の長さの 0.75 s からリリースまでの,投擲フォームとトルクの時系列が類似していた.よって,身長1.72 m の人間に基づいた腕の長さと身長1.90 m の人間に基づいた腕の長さによる遠投をするための投擲フォームは,ともに運動連鎖による肩関節の回転を重要視した戦略が考えられる.身長1.90 m の人間に基づいた腕の長さの際の投擲開始からの挙動は,モーメントアームを小さくするために肘を屈曲し,そこから伸展することで腕を振り上げるためのエネルギーを大きくしていると考えられる.
\section{結言}
本研究では投擲物の重さや身体のパラメータに応じた投擲フォームを導出・比較することで,パラメータに応じた投擲フォーム戦略を考察した.まず,剛体1リンクモデルでの強化学習を用いたリンク速度最適化シミュレーションにより,自作した強化学習シミュレータと手法の有用性を確認した.その後,腕に見立てた2次元剛体2リンクモデルへと拡張し,投擲物の重さや腕の長さに応じた投擲フォーム戦略の考察を行い,パラメータに応じた投擲フォーム戦略を考察した.その後,より人間に近い3次元の腕モデルへと拡張し,投擲物の重さや腕の長さに応じた投擲フォーム戦略について考察を行った.投擲物の重さに応じた遠投をするための投擲フォーム戦略の比較の結果,腕の押し出し度合いによる戦略の違いがみられた.また,腕の長さに応じた遠投をするための投擲フォーム戦略の比較の結果,投擲開始から慣性モーメントの影響により挙動に違いが生じたが,ともに運動連鎖による肩関節の回転を重要視した戦略がみられた.また,本研究の手法は投擲フォーム戦略を考察することにおいて有用であることを示した.\\
 今後の展望として,深層強化学習を用いたトルクの連続値入力による学習,変更する投擲物の重さや身体のパラメータの種類の増加がある.また,本研究では2次元から3次元の拡張で新たに追加された要素によって新たな戦略もみられた.そのため,体幹の追加や全身モデルでの学習による投擲フォーム戦略の考察により,全身の運動連鎖の傾向等,より実際の人間に近い投擲フォーム戦略の考察が可能であると考えられる.加えてさまざまなスポーツに応じたルール制約を設けた学習により,実際の競技や個人に応じた投擲フォーム戦略がみられると考えられる.
\figt{5.23.eps}{width=1.0\hsize}{Transition of shoulder and elbow angular velocity from start to release of 3D Rigid 2 Links(left:1.72 m tall human,right:1.90 m tall human)}
%% \begin{thebibliography}{99}
%% \small
%%  \setlength{\kanjiskip}{0.0zw plus.01zw} %
%%  \setlength{\baselineskip}{9pt}        %
%%  \setlength{\itemsep}{0.2pt}             %
%%  \setlength{\lineskip}{0pt}              %
%%  \setlength{\normallineskip}{0.2pt}      %


%% \bibitem{hogege} 川村マサキ,
%% ほげの可能性と適用限界に関する実験的研究,日本ほげ学会ほげ工学部門講演会,(2010).


%% \bibitem{hohoge} 本堂貴敏,
%% ほげの力学,(2006),pp.11--43,ほげ出版.

%% \end{thebibliography}



\section{結論および今後の展望}
本研究では投擲物の重さや身体のパラメータに応じた投擲フォームを導出・比較することで,パラメータに応じた投擲フォーム戦略を考察した.まず,剛体1リンクモデルでの強化学習を用いたリンク速度最適化シミュレーションにより,自作した強化学習シミュレータと手法の有用性を確認した.その後,腕に見立てた2次元剛体2リンクモデルへと拡張し,投擲物の重さや腕の長さに応じた投擲フォーム戦略の考察を行い,パラメータに応じた投擲フォーム戦略を考察した.その後,より人間に近い3次元の腕モデルへと拡張し,投擲物の重さや腕の長さに応じた投擲フォーム戦略について考察を行った.投擲物の重さに応じた遠投をするための投擲フォーム戦略の比較の結果,腕の押し出し度合いによる戦略の違いがみられた.また,腕の長さに応じた遠投をするための投擲フォーム戦略の比較の結果,投擲開始から慣性モーメントの影響により挙動に違いが生じたが,ともに運動連鎖による肩関節の回転を重要視した戦略がみられた.また,本研究の手法は投擲フォーム戦略を考察することにおいて有用であることを示した.\\
 今後の展望として,深層強化学習を用いたトルクの連続値入力による学習,変更する投擲物の重さや身体のパラメータの種類の増加がある.また,本研究では2次元から3次元の拡張で新たに追加された要素によって新たな戦略もみられた.そのため,体幹の追加や全身モデルでの学習による投擲フォーム戦略の考察により,全身の運動連鎖の傾向等,より実際の人間に近い投擲フォーム戦略の考察が可能であると考えられる.加えてさまざまなスポーツに応じたルール制約を設けた学習により,実際の競技や個人に応じた投擲フォーム戦略がみられると考えられる.

%  今後の展望としては,本論文の検証内容を我々が開発した
% 小型の実機にて検証実験を行う.
% 加えて,積載物の転がらず滑らない条件を一般化し,それを満たす歩行方法を3次元LIPMを基に定式化する.
% そして,シミュレーションと実機を用いて検証を行う.
% 最終的には,人が乗れるサイズの実機を開発し,積載物を落とさない搬送を実現する.


\footnotesize

\bibliography{reference} % 拡張子を外したbibファイル
\bibliographystyle{junsrt} % 参考文献出力のスタイル

\normalsize
\end{document}
