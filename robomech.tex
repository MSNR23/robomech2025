\section{緒言}
投擲動作を行うスポーツは数多く存在するが,野球と砲丸投げのように競技によって投擲フォームは異なり,さらに同一競技内であっても個人によって投擲フォームは異なる.競技や個人によって投擲フォームが異なる要因として,投擲物や身体といった投擲フォームに関連するパラメータの違いが挙げられる.これまで投擲フォームに関する研究例として,眞田の野球におけるオーバーハンドスローとサイドハンドスローの球速の比較\cite{sanada},黒松らの砲丸投げグライド投法における投擲フォーム改善前後の飛距離の比較\cite{kuromatsu}などがある.また,投擲に関する総合性能の研究例として,蔭山らの野球における体格や背筋力と投球速度の関係\cite{kageyama},坪井らの砲丸投げにおける投射速度・投射角と飛距離の関係\cite{tsuboi}などがある.スポーツにおいて総合性能向上は最も重要な要素の一つである.これらの研究はある一つの競技に特定した研究である.しかし,さまざまな投擲フォームがどのような戦略の基で成立しているのかに関する汎用的な理論は確立されていない.本研究ではシミュレーションにおいてさまざまなパラメータに応じた投擲フォームを導出・比較することで,さまざまな投擲フォームの戦略を検討・考察・議論することを目的とする.\\
\section{強化学習を用いた投擲フォーム導出}
投擲フォームの導出方法について,本研究では強化学習による最適化手法を採用した.投擲フォームは時々刻々と全身の運動連鎖\cite{burkhart}によるダイナミクスが変化するため,明示的な解を求めることは困難である.運動連鎖とは,ある関節の動作が隣接する関節に影響を与え,運動エネルギーを伝達していく運動効果である.強化学習は明示的な解がなく詳細なパラメータ設定が求められる投擲フォーム導出において有効な手段である.
\subsection{強化学習手法}
本研究では,強化学習の手法の1つであるQ学習\cite{watkins}を採用した.Q学習において,Q値は\equref{1}で更新する.
\begin{eqnarray}
  \equlabel{1}
  Q(s,a)=(1-\alpha)Q(s,a)+\alpha(r+\gamma \mathrm{max}Q(s',a'))
\end{eqnarray}
\equref{1}において,$\alpha$は学習率,$\gamma$は割引率であり,範囲はそれぞれ0$\leq$$\alpha$$\leq$1,0$\leq$$\gamma$$\leq$1である.また,$Q(s,a)$は状態が$s$における行動$a$の行動価値関数である.学習率はQ値の更新の大きさを決定するパラメータであり,割引率は将来の価値をどれだけ割り引くのかを決定するパラメータである.
\subsection{行動選択方法}
本研究では,行動選択方法として$\varepsilon$-greedy法\cite{greedy}を採用した.$\varepsilon$-greedy法では,$\varepsilon$(0$\leq$$\varepsilon$$\leq$1)の確率で全ての行動からランダムに行動を選択し,1-$\varepsilon$の確率でルールの価値が最も高い行動を選択する.\\
\section{2次元剛体2リンクによるパラメータに応じた投擲フォームの導出・比較と戦略の考察}
\subsection{動力学モデル}
本章の検証で用いた動力学モデルを\figref{4.0.eps}に示す.人間の腕を肩関節1自由度と肘関節1自由度の計2自由度から構成される2次元剛体2リンクとしてモデル化した.前腕リンクは前腕と手と投擲物の要素を合わせたものとした.$l_{1}$は上腕リンク長さ,$l_{2}$は前腕リンク長さ,$l_{g1}$は上腕リンクの重心までの長さ,$l_{g2}$は前腕リンクの重心までの長さ,$m_{1}$は上腕リンク重さ,$m_{2}$は前腕リンク重さ,$I_{1}$は上腕リンクの重心周りの慣性モーメント,$I_{2}$は前腕リンクの重心周りの慣性モーメント,$\tau_{1}$は肩関節にかかるトルク,$\tau_{2}$は肘関節にかかるトルク,$\theta_{1}$は上腕リンク角度,$\theta_{2}$は前腕リンク角度である.また,重力加速度は$g$である.ラグランジュの方法を用いて導いた2次元剛体2リンクの運動方程式は,次の通りである.
\begin{eqnarray}
  \equlabel{5}
  M(\theta)\ddot{\theta} + h(\theta, \dot{\theta}) + g(\theta) + b(\dot{\theta}) = \tau
\end{eqnarray}
このとき,$\theta$=${[\theta_{1} \quad \theta_{2}]}^T$,$\tau$=${[\tau_{1} \quad \tau_{2}]}^T$とし,$M(\theta)$は慣性行列,$h(\theta, \dot{\theta})$はコリオリ力の項,$g(\theta)$は重力項,$b(\dot{\theta})$は粘性項である.\\
 導出した運動方程式をRunge-Kutta法\cite{runge-kutta}により数値積分し運動学を解くことで,2次元剛体2リンクの角度や角速度を計算した.
\figt{4.3.eps}{width=0.7\hsize}{Throwing form from start to release of 2D Rigid 2 Links(0.14 kg thrown object,1.72 m tall human)}
\figt{4.4.eps}{width=0.7\hsize}{Throwing form from start to release of 2D Rigid 2 Links(7.24 kg thrown object,1.72 m tall human)}
\subsection{強化学習の設定}
状態について,状態変数は4つとし,肩関節の角度$\theta_{1}$,角速度$\dot{\theta}_{1}$,肘関節の角度$\theta_{2}$,角速度$\dot{\theta}_{2}$とした.\\
各関節の角度の可動範囲は,$-180$ deg $\le$ $\theta_{1}$ $\le$ $180$ deg,$0$ $\le$ $\theta_{2}$ $\le$ $150$ deg\cite{range}とした.
また,各リンクの角速度については,$-10.0$ m/s $\le$ $\dot{\theta}_{1}$ $\le$ $10.0$ m/s,$-10.0$ m/s $\le$ $\dot{\theta}_{2}$ $\le$ $10.0$ m/sとした.
分割数は各状態4分割であり,全ての状態を$4^{4}$=256通りで表すことができる.\\
行動は,全9通りに設定した.肩関節にかかるトルクを正,0,負の3通り,同様に肘関節にかかるトルクも正,0,負の3通りとした.
これにより,Qテーブルは$256 \times 9=2304$通りで表すことができる.\\
報酬について,報酬の設計を\equref{6}に示す.$\theta_{v}$は投射角度,$v_{2}$は手先速度,$CE$は肩関節と肘関節の累積消費エネルギーの合計である.$scale$は累積消費エネルギーの値を調整するものである.
\begin{eqnarray}
  \equlabel{6}
  \mathrm{Reward} = \theta_{v} \times v_{2} - scale \times CE
\end{eqnarray}
その他の設定について,\equref{1}における学習率$\alpha$を$0.1$,割引率$\gamma$を$0.9$に設定した.1タイムステップは$0.005$ sであり,1エピソード内のステップ数は2000,すなわち,$10$ sとした.エピソード数に関しては,学習が収束するまで際限なく行うものとした.
\figt{4.5.eps}{width=0.8\hsize}{Transition of shoulder and elbow torque from start to release of 2D Rigid 2 Links(left: 0.14 kg thrown object,right: 7.24 kg thrown object)}
\figt{4.6.eps}{width=0.8\hsize}{Transition of shoulder and elbow angular velocity from start to release of 2D Rigid 2 Links(left: 0.14 kg thrown object,right: 7.24 kg thrown object)}
\subsection{投擲物の重さに応じた投擲フォームの結果・考察}
\subseclabel{4.3}
身体のサイズは,身長 1.72 m,体重 70 kgの人間の各部位のサイズとした.投擲物は,野球の硬式球と砲丸の重さを参考に,$0.14$ kg\cite{horiuchi} と$7.24$ kg\cite{haq} とした.\\
 重さ0.14 kg の投擲物と重さ7.24 kg の投擲物での学習の際に用いる各パラメータを\tabref{1}に示す\cite{irving}.重力加速度$g$を9.8 $\mathrm{m/s^{2}}$とする.肩関節に与えるトルクについて,重さ0.14 kg の際は$-16$ N$\cdot$m,0,16 N$\cdot$m,重さ7.24 kg の際は$-47$ N$\cdot$m,0,47 N$\cdot$mとした.また,肘関節に与えるトルクについて,重さ0.14 kg の際は$-10$ N$\cdot$m,0,10 N$\cdot$m,重さ7.24 kg の際は$-41$ N$\cdot$m,0,41 N$\cdot$mとした.さらに肘関節のトルクの設定において,可動範囲の端である$\theta_{2} = 0$ または $\theta_{2} = 150$ degでは,可動範囲内に戻ろうとする力がはたらく.これを考慮し,可動範囲の端では肘関節が可動範囲内に戻る方向に20 N$\cdot$m のトルクを加算する設定とした.各関節の粘性係数について,肩関節の粘性係数$b_{1}$は0.05,肘関節の粘性係数$b_{2}$は0.01とした.
以上の設定により学習を行い,投擲物の重さに応じた遠投をするための投擲フォームの比較を行った.\\
 $\theta_{1}$,$\theta_{2}$,$\dot{\theta}_{1}$,$\dot{\theta}_{2}$の初期状態をいずれも0とした際の投擲物の重さに応じた遠投をするための投擲フォーム戦略の結果・考察について述べる.
\figref{4.3.eps}は重さ0.14 kg の投擲物を遠投するための投擲フォーム,\figref{4.4.eps}は重さ7.24 kg の投擲物を遠投するための投擲フォームである.投擲フォームは投擲開始からリリースまでとし,リリース時の姿勢は黒の太枠で囲まれたものである.図中の青リンク,橙リンクがそれぞれ上腕リンク,前腕リンクに対応し,赤線が手先の軌跡を示す.リリース瞬間の手先速度は,重さ0.14 kg の投擲物の際が10.24 m/s,重さ7.24 kg の投擲物の際が5.46 m/sであった.また,投射角は,重さ0.14 kg の投擲物の際が43.76 deg,重さ7.24 kg の投擲物の際が47.81 deg であった.\\
\begin{table}[tb]
  \begin{center}
    \caption{Parameters which are used for 2D Rigid 2 Links simulation(1.72 m vs 1.90 m)}
    \tablabel{2}
    \scalebox{0.7}[0.7]{
    \begin{tabular}{c|c|c|c}
      \hline
      Parameters & Unit & Values of 1.72 m & Values of 1.90 m \\
      \hline
      $l_{1}$ & m & 0.32 & 0.35 \\
      $l_{2}$ & m & 0.44 & 0.48 \\
      $l_{g1}$ & m & 0.16 & 0.18 \\
      $l_{g2}$ & m & 0.21 & 0.23 \\
      $m_{1}$ & kg & 1.96 & 1.96\\
      $m_{2}$ & kg & 1.68 & 1.68\\
      $I_{1}$ & kg$\cdot$$\mathrm{m}^2$ & 0.017 & 0.018 \\
      $I_{2}$ & kg$\cdot$$\mathrm{m}^2$ & 0.020 & 0.099 \\
      \hline
    \end{tabular}
    }
  \end{center}
\end{table}
\figt{4.8.eps}{width=0.7\hsize}{Throwing form from start to release of 2D Rigid 2 Links(0.14 kg thrown object,1.90 m tall human)}
 \figref{4.5.eps}は各関節のトルクの推移,\figref{4.6.eps}は各関節の角速度の推移である.\figref{4.5.eps},\figref{4.6.eps}の左図が重さ0.14 kg の投擲物の際で,右図が重さ7.24 kg の投擲物の際の各時系列である.トルクの時系列より,重さ0.14 kg の投擲物を遠投するための投擲フォームは,肩関節と肘関節の運動連鎖を活かした投擲フォーム戦略の傾向がみられた.約0.13 s で肩関節のトルクが腕の回転とは逆方向に,肘関節のトルクが腕の回転と同方向に生じた.これは肩関節のエネルギーを肘関節に伝達することで,より手先方向のエネルギーを大きくしていると考えられる.また,角速度の推移より,リリース時には肩関節と肘関節の角速度が逆転していることから最後は前腕の振りが重要であると考えられる.一方,重さ7.24 kg の投擲物を遠投するための投擲フォームは,肩関節による影響が大きい投擲フォームの傾向がみられた.肩関節のトルクが0となる瞬間が数回生じ,肘関節のトルクは肘を屈曲させる方向に生じていた.これは,モーメントアームを小さくすることで肩関節のエネルギー消費を抑制するためであると考えられる.また,角速度の推移より,リリースの瞬間も肩関節の角速度が肘関節の角速度より大きい.すなわち,肩関節の影響が重さ0.14 kg の投擲物を遠投する際よりも大きいと考えられる.
\subsection{腕の長さに応じた投擲フォームの結果・考察}
\subseclabel{4.4}
身体のサイズは,身長1.72 m,体重70 kgの人間と,身長1.90 m,体重70 kg の各部位のサイズとした.投擲物は,野球の硬式球を参考に,0.14 kgとした.\\
 身長1.72 m の人間に基づいた腕の長さでの投擲の際と身長1.90 m の人間に基づいた腕の長さでの学習の際に用いる各パラメータを\tabref{2}に示す.重力加速度$g$を9.8 $\mathrm{m/s^{2}}$とする.肩関節に与えるトルクについて,身長1.72 m の人間に基づいた腕の長さの際は$-16$ N$\cdot$m,0,16 N$\cdot$m,身長1.90 m の人間に基づいた腕の長さの際は$-18$ N$\cdot$m,0,18 N$\cdot$mとした.また,肘関節に与えるトルクについて,身長1.72 m の人間に基づいた腕の長さの際は$-10$ N$\cdot$m,0,10 N$\cdot$m,身長1.90 m の人間に基づいた腕の長さの際は$-11$ N$\cdot$m,0,11 N$\cdot$mとした.さらに肘関節の可動範囲の端のトルク,各関節の粘性係数の設定においては,投擲物の重さに応じた比較の際と同様の設定とした.
以上の設定により学習を行い,腕の長さに応じた遠投をするための投擲フォームの比較を行った.\\
\figt{4.9.eps}{width=0.8\hsize}{Transition of shoulder and elbow torque from start to release of 2D Rigid 2 Links(left:1.72 m tall human,right:1.90 m tall human)}
\figt{4.10.eps}{width=0.8\hsize}{Transition of shoulder and elbow angular velocity from start to release of 2D Rigid 2 Links(left:1.72 m tall human,right:1.90 m tall human)}
 $\theta_{1}$,$\theta_{2}$,$\dot{\theta}_{1}$,$\dot{\theta}_{2}$の初期状態をいずれも0とした際の腕の長さに応じた遠投をするための投擲フォーム戦略の結果・考察について述べる.
\figref{4.3.eps}は身長1.72 m の人間に基づいた腕の長さによる遠投をするための投擲フォーム,\figref{4.8.eps}は身長1.90 m の人間に基づいた腕の長さによる遠投をするための投擲フォームである.リリース瞬間の手先速度は,身長1.72 m の人間に基づいた腕の長さの際が10.24 m/s,身長1.90 m の人間に基づいた腕の長さの際が13.95 m/sであった.また,投射角は,身長1.72 m の人間に基づいた腕の長さの際が43.76 deg,身長1.90 m の人間に基づいた腕の長さの際が51.18 deg であった.\\
 \figref{4.9.eps}は各関節のトルクの推移,\figref{4.10.eps}は各関節の角速度の推移である.\figref{4.9.eps},\figref{4.10.eps}の左図が身長1.72 m の人間に基づいた腕の長さの際で,右図が身長1.90 m の人間に基づいた腕の長さの際の各時系列である.トルクの時系列より,身長1.90 m の人間に基づいた腕の長さによる遠投をするための投擲フォームは,身長1.72 m の人間に基づいた腕の長さによる遠投をするための投擲フォームと同様に運動連鎖を活かした投擲フォーム戦略の傾向がみられた.しかし,身長1.90 m の人間に基づいた腕の長さによる遠投をするための投擲フォームは肘関節のトルクを腕の回転方向とは反対の,肘関節を屈曲させる方向へ生じさせている.これはモーメントアームを小さくしてエネルギー消費を少なく腕を回転させることを優先したと考えられる.また,腕の長さに問わずリリース時には肘関節の角速度が肩関節の角速度を上回ることから,リリース直前に運動連鎖によってより肘関節の角速度を大きくしようとすると考えられる.
\section{3次元剛体2リンクによるパラメータに応じた投擲フォームの導出・比較と戦略の考察}
\subsection{動力学モデル}
本章で用いた動力学モデルを\figref{5.1.eps}に示す.物理エンジンMuJoCo\cite{mujoco}に標準搭載されているhumanoidモデル「Unitree G1」\cite{unitreeg1}を改変し,人間の腕を肩関節3自由度と肘関節1自由度の計4自由度から構成される3次元剛体2リンクとしてモデル化した.なお,\figref{5.1.eps}において,肩関節から肘関節までを上腕リンク,肘関節から手先までを前腕リンクとし,手首や指の自由度は0とした.また,体幹リンクも自由度0とし,世界座標に固定した.運動方程式はMuJoCoで内部的に解いた.Runge-Kutta法により数値積分し運動学を解くことで,3次元剛体2リンクの角度や角速度を計算した.
\figt{5.1.eps}{width=0.5\hsize}{3D Rigid 2 Links(stabilization of the trunk)}
\subsection{強化学習の設定}
状態について,状態変数は8つとし,肩関節ピッチ軸周りの角度$\theta_{p}$,角速度$\dot{\theta}_{p}$,肩関節ロール軸周りの角度$\theta_{r}$,角速度$\dot{\theta}_{r}$,肩関節ヨー軸周りの角度$\theta_{y}$,角速度$\dot{\theta}_{y}$,肘関節の角度$\theta_{e}$,角速度$\dot{\theta}_{e}$とした.
角度について,それぞれの可動範囲の角度は,肩関節ピッチ軸周りが$-135$ deg $\leq\theta_{p}\leq45$  deg,肩関節ロール軸周りが$-135$ deg $\leq\theta_{r}\leq-35$ deg,肩関節ヨー軸周りが$-150$ deg $\leq\theta_{y}\leq180$ deg,肘関節が$-20$ deg $\leq\theta_{e}\leq90$ deg とした.しかし,人間の各関節は互いに干渉するため,組み合わせ次第では人間が不可能な姿勢となる.そのため,スプライン補間\cite{spline}を用いて肩関節ピッチ軸周りとロール軸周りの角度によって肩関節ヨー軸周りの可動範囲が変動するように設定した.また,各関節の角速度については,いずれも$-10.0$ m/s $\le$ $\dot{\theta}_{i}$ $\le$ 10.0 m/s($i = p,r,y,e$)とした.
分割数は各角度が5分割,各角速度が2分割であり,全ての状態を$5^{4}\times 2^{4}=10000$通りで表すことができる.\\
行動は,全625通りに設定した.肩の各関節にかかるトルクを正2通り,0,負2通りの計5通り,同様に肘関節にかかるトルクも正2通り,0,負2通りの計5通りとした.
これにより,Qテーブルは$10000 \times 625=6250000$通りで表すことができる.\\
報酬は投擲物の飛距離とし,\equref{7}で計算した.投擲物のモデル化は行っていないため,投射中の投擲物に生じる空気抵抗等は考慮しないものとする.
\begin{eqnarray}
  \equlabel{7}
  D = v_{syn} \cdot \cos\theta_{v} \cdot t
\end{eqnarray}
飛距離の計算にあたり,$v_{syn}$はピッチ軸方向,ロール軸方向,ヨー軸方向の3方向成分を合成した手先速度,$\theta_{v}$は投射角度,$t$は投射時間とした.
また,2次元の際と同様に,罰則として累積消費エネルギー$CE$を採用した.累積消費エネルギーは肩関節3軸周りと肘関節の消費エネルギーの合計とした.以上より,報酬の設計は,\equref{8}とした.
\begin{eqnarray}
  \equlabel{8}
  R = D - scale \times CE
\end{eqnarray}
 その他の設定について,\equref{1}における学習率$\alpha$を0.1,割引率$\gamma$を0.9に設定した.1タイムステップは0.001 sであり,1エピソード内のステップ数は3000,すなわち,3 sとした.エピソード数に関しては,10000エピソードの中で最も報酬が高いエピソードを採用した.
\subsection{投擲物の重さに応じた投擲フォームの結果・考察}
身体のサイズ,投擲物,各関節に与えるトルクと粘性係数以外のパラメータは\subsecref{4.3}の2次元剛体2リンクの際と同様とした.肩関節3軸周りに与えるトルクについては,$-40$ N$\cdot$m,$-20$ N$\cdot$m,0,20 N$\cdot$m,40 N$\cdot$mから,肘関節に与えるトルクについては$-30$ N$\cdot$m,$-15$ N$\cdot$m,0,15 N$\cdot$m,30 N$\cdot$mから\equref{4}に基づいて選択したが,肩関節ヨー軸周りのトルクに関しては,スプライン補間によって設定した可動範囲に基づき,トルクも補完する設定とした.各関節の粘性係数について,肩関節ピッチ軸周りの粘性係数$b_{p}$は1.0,肩関節ロール軸周りの粘性係数$b_{r}$は0.8,肩関節ヨー軸周りの粘性係数$b_{y}$は0.5,肘関節の粘性係数$b_{e}$は0.2とした.
以上の設定により学習を行い,投擲物の重さに応じた遠投をするための投擲フォームの比較を行った.\\
 初期状態について,$\theta_{p}$,$\theta_{r}$,$\theta_{y}$,$\theta_{e}$はランダム値,$\dot{\theta}_{p}$,$\dot{\theta}_{r}$,$\dot{\theta}_{y}$,$\dot{\theta}_{e}$は0とした.これにより,最も報酬が高くなる際の初期姿勢を導出することができる.投擲物に応じた遠投するための投擲フォーム戦略の結果・考察について述べる.\figref{5.11.eps}はピッチ軸に垂直な面から見た際の重さ0.14 kg の投擲物を遠投するための投擲フォーム,\figref{5.13.eps}は重さ7.24 kg の投擲物を遠投するための投擲フォームである.リリース瞬間は1エピソードの中で最も報酬が高いステップとし,投擲フォームは投擲開始からリリースまでとした.重さ0.14 kg の投擲物を遠投するための投擲フォームのリリース瞬間は0.489 s で,投擲物の飛距離が70.17 m,重さ7.24 kg の投擲物を遠投するための投擲フォームのリリース瞬間は0.447 s で,投擲物の飛距離が6.42 mであった.\\
\figt{5.11.eps}{width=0.8\hsize}{Throwing form from start to release of 3D Rigid 2 Links from a plane perpendicular to the pitch axis(0.14 kg thrown object,1.72 m tall human)}
\figt{5.13.eps}{width=0.8\hsize}{Throwing form from start to release of 3D Rigid 2 Links from a plane perpendicular to the pitch axis(7.24 kg thrown object,1.72 m tall human)}
 \figref{5.7.eps}は各関節のトルクの推移,\figref{5.15.eps}は肩を基準とした際の手先高さの推移である.\figref{5.7.eps},\figref{5.15.eps}において,左図は重さ0.14 kg の投擲物の際で,右図は重さ7.24 kg の投擲物の際の各時系列である.これらより,投擲物の重さに応じた投擲フォームの比較によりみられる戦略の違いは,「腕の押し出し度合い」であると考えられる.\figref{5.15.eps}において,重さ0.14 kg の投擲物の際はリリース前に一度手先が下がってから一気に高くなるが,重さ7.24 kg の投擲物の際はほぼ手先高さに変化がみられない.この波形の直線度が高いほど腕の押し出し度合いが高い.また,0.35 s あたりから,運動連鎖よって肩関節ピッチ軸周りの関節が持つエネルギーをヨー軸周りの関節に伝達している傾向がみられた.よって,重さ0.14 kg の投擲物を遠投するための投擲フォームは,運動連鎖による肩関節の回転を重要視した戦略が考えられる.一方,重さ7.24 kg の投擲物を遠投するための投擲フォームは,運動連鎖による前腕へのエネルギーの伝達により,肘を伸展させるエネルギーを大きくする傾向がみられた.よって,肘の伸展や前腕を重要視した腕の押し出し度合いの高い戦略が考えられる.
\figt{5.7.eps}{width=0.8\hsize}{Transition of shoulder and elbow torque from start to release of 3D Rigid 2 Links(left: 0.14 kg thrown object,right: 7.24 kg thrown object)}
\figt{5.15.eps}{width=0.8\hsize}{Transition of hand height relative shoulder from start to release of 3D Rigid 2 Links(left: 0.14 kg thrown object,right: 7.24 kg thrown object)}
\subsection{腕の長さに応じた投擲フォームの結果・考察}
身体のサイズ,投擲物,各関節に与えるトルクと粘性係数以外のパラメータは\subsecref{4.4}の2次元剛体2リンクの際と同様とした.また,各関節に与えるトルクと各関節の粘性係数の設定については,投擲物の重さによる比較の際と同様とした.
以上の設定により学習を行い,腕の長さに応じた遠投をするための投擲フォームの比較を行った.\\
 初期状態について,投擲物の重さによる投擲フォームの比較の際と同様の設定とした.腕の長さに応じた遠投をするための投擲フォーム戦略の結果・考察について述べる.
\figref{5.11.eps}はピッチ軸に垂直な面から見た際の身長1.72 m の人間に基づいた腕の長さによる遠投をするための投擲フォーム,\figref{5.16.eps}は身長1.90 m の人間に基づいた腕の長さによる遠投をするための投擲フォームである.リリース瞬間は1エピソードの中で最も報酬が高いステップとし,投擲フォームは投擲開始からリリースまでとした.身長1.72 m の人間に基づいた腕の長さによる遠投をするための投擲フォームのリリース瞬間は0.489 s で,投擲物の飛距離が70.17 m,身長1.90 m の人間に基づいた腕の長さによる遠投をするための投擲フォームのリリース瞬間は1.109 s で,投擲物の飛距離が92.16 mであった.
\figref{5.8.eps}は各関節のトルクの推移,\figref{5.23.eps}は各関節の角速度の推移である.\figref{5.8.eps},\figref{5.23.eps}において,左図は身長1.72 m の人間に基づいた腕の長さの際の,右図は身長1.90 m の人間に基づいた腕の長さの際の各時系列である.\\
 比較の結果,身長1.72 m の人間に基づいた腕の長さの 0.28 s からリリースまでと身長1.90 m の人間に基づいた腕の長さの 0.75 s からリリースまでの,投擲フォームとトルクの時系列が類似していた.よって,身長1.72 m の人間に基づいた腕の長さと身長1.90 m の人間に基づいた腕の長さによる遠投をするための投擲フォームは,ともに運動連鎖による肩関節の回転を重要視した戦略が考えられる.身長1.90 m の人間に基づいた腕の長さの際の投擲開始からの挙動は,モーメントアームを小さくするために肘を屈曲し,そこから伸展することで腕を振り上げるためのエネルギーを大きくしていると考えられる.
\figt{5.16.eps}{width=0.8\hsize}{Throwing form from start to release of 3D Rigid 2 Links from a plane perpendicular to the pitch axis(0.14 kg thrown object,1.90 m tall human)}
\figt{5.8.eps}{width=0.8\hsize}{Transition of shoulder and elbow torque of 3D Rigid 2 Links from start to release (left:1.72 m tall human,right:1.90 m tall human)}
\section{結論および今後の展望}
本研究では投擲物の重さや身体のパラメータに応じた投擲フォームを導出・比較することで,パラメータに応じた投擲フォーム戦略を考察した.まず,剛体1リンクモデルでの強化学習を用いたリンク速度最適化シミュレーションにより,自作した強化学習シミュレータと手法の有用性を確認した.その後,腕に見立てた2次元剛体2リンクモデルへと拡張し,投擲物の重さや腕の長さに応じた投擲フォーム戦略の考察を行い,パラメータに応じた投擲フォーム戦略を考察した.その後,より人間に近い3次元の腕モデルへと拡張し,投擲物の重さや腕の長さに応じた投擲フォーム戦略について考察を行った.投擲物の重さに応じた遠投をするための投擲フォーム戦略の比較の結果,腕の押し出し度合いによる戦略の違いがみられた.また,腕の長さに応じた遠投をするための投擲フォーム戦略の比較の結果,投擲開始から慣性モーメントの影響により挙動に違いが生じたが,ともに運動連鎖による肩関節の回転を重要視した戦略がみられた.また,本研究の手法は投擲フォーム戦略を考察することにおいて有用であることを示した.\\
 今後の展望として,深層強化学習を用いたトルクの連続値入力による学習,変更する投擲物の重さや身体のパラメータの種類の増加がある.また,本研究では2次元から3次元の拡張で新たに追加された要素によって新たな戦略もみられた.そのため,体幹の追加や全身モデルでの学習による投擲フォーム戦略の考察により,全身の運動連鎖の傾向等,より実際の人間に近い投擲フォーム戦略の考察が可能であると考えられる.加えてさまざまなスポーツに応じたルール制約を設けた学習により,実際の競技や個人に応じた投擲フォーム戦略がみられると考えられる.
\figt{5.23.eps}{width=0.8\hsize}{Transition of shoulder and elbow angular velocity from start to release of 3D Rigid 2 Links(left:1.72 m tall human,right:1.90 m tall human)}
%% \begin{thebibliography}{99}
%% \small
%%  \setlength{\kanjiskip}{0.0zw plus.01zw} %
%%  \setlength{\baselineskip}{9pt}        %
%%  \setlength{\itemsep}{0.2pt}             %
%%  \setlength{\lineskip}{0pt}              %
%%  \setlength{\normallineskip}{0.2pt}      %


%% \bibitem{hogege} 川村マサキ,
%% ほげの可能性と適用限界に関する実験的研究,日本ほげ学会ほげ工学部門講演会,(2010).


%% \bibitem{hohoge} 本堂貴敏,
%% ほげの力学,(2006),pp.11--43,ほげ出版.

%% \end{thebibliography}

