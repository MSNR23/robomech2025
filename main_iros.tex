\documentclass[letterpaper, 10 pt, conference,dvipdfmx]{ieeeconf}  % Comment this line out if you need a4paper

%\documentclass[a4paper, 10pt, conference]{ieeeconf}      % Use this line for a4 paper
% \include{my_layout_grad}
% \usepackage{jtygm}
\usepackage{ikuo}%%便利コマンド集.
\usepackage{siunitx}
\usepackage{jtygm}
\usepackage{bm}
\usepackage{balance}
\usepackage{amssymb}
\usepackage{here}
\usepackage{amsmath}

\newcommand{\FIGDIR}{./fig}
\newcommand{\ctext}[1]{\raise0.1ex\hbox{\textcircled{\scriptsize{#1}}}}


\IEEEoverridecommandlockouts                              % This command is only needed if 
                                                          % you want to use the \thanks command

\overrideIEEEmargins                                      % Needed to meet printer requirements.

%In case you encounter the following error:
%Error 1010 The PDF file may be corrupt (unable to open PDF file) OR
%Error 1000 An error occurred while parsing a contents stream. Unable to analyze the PDF file.
%This is a known problem with pdfLaTeX conversion filter. The file cannot be opened with acrobat reader
%Please use one of the alternatives below to circumvent this error by uncommenting one or the other
%\pdfobjcompresslevel=0
%\pdfminorversion=4

% See the \addtolength command later in the file to balance the column lengths
% on the last page of the document

% The following packages can be found on http:\\www.ctan.org
\usepackage{graphics} % for pdf, bitmapped graphics files
\usepackage{epsfig} % for postscript graphics files
%\usepackage{mathptmx} % assumes new font selection scheme installed
%\usepackage{times} % assumes new font selection scheme installed
%\usepackage{amsmath} % assumes amsmath package installed
%\usepackage{amssymb}  % assumes amsmath package installed

\title{\LARGE \bf
二脚搬送ロボットにおける積載物転倒・滑り条件とLIPMでの抑制方法の検討
% Preparation of Papers for IEEE Sponsored Conferences \& Symposia*
}
% 二脚搬送ロボットにおける積載物転倒・滑り条件とLIPMでの回避方法の検討

\author{Kimikage Kanno$^{1}$ and Ikuo Mizuuchi$^{1}$% <-this % stops a space
\thanks{$^{1}$Department of Mechanical Systems Engineering, Tokyo University of Agriculture and Technology, 2-24-16, Naka-cho, Koganei-city, Tokyo, Japan
        {\tt\small \{kanno,ikuo\}@mizuuchi.lab.tuat.ac.jp}}%
}
% \author{Albert Author$^{1}$ and Bernard D. Researcher$^{2}$% <-this % stops a space
% \thanks{*This work was not supported by any organization}% <-this % stops a space
% \thanks{$^{1}$Albert Author is with Faculty of Electrical Engineering, Mathematics and Computer Science,
%         University of Twente, 7500 AE Enschede, The Netherlands
%         {\tt\small albert.author@papercept.net}}%
% \thanks{$^{2}$Bernard D. Researcheris with the Department of Electrical Engineering, Wright State University,
%         Dayton, OH 45435, USA
%         {\tt\small b.d.researcher@ieee.org}}%
% }


\begin{document}



\maketitle
\thispagestyle{empty}
\pagestyle{empty}


%%%%%%%%%%%%%%%%%%%%%%%%%%%%%%%%%%%%%%%%%%%%%%%%%%%%%%%%%%%%%%%%%%%%%%%%%%%%%%%%
\begin{abstract} 
本論文では二足歩行ロボットに物や人を搭載して搬送する場合を想定し,
積載物が転がらず滑らないロボットボディ加速度の範囲を定式化した.
物を搬送するロボットにおいて,積載物を落とさずに運ぶことは必須である.
そのためには積載物の転がりと滑りを抑制することが重要である.
そして,積載物が転がらず滑らない歩幅,重心高さをLIPMを基に定式化した.
そして,シミュレーション上にて定式化した歩幅,重心高さの範囲内であれば
歩行中の積載物の転がりを防止し,滑りを軽減できることを確認した.\\        
% This electronic document is a live template. The various components of your paper [title, text, heads, etc.] are already defined on the style sheet, as illustrated by the portions given in this document.
\end{abstract}
% 物を搬送するロボットが二足歩行となることで,段差を気にせずに移動でき,
% 物を搬送できる範囲が広がる.物を搬送するロボットにおいて,積載物の
% 落とさず運ぶことは必須である.そのためには転がりと滑りを抑制することは
% 重要である.
% 本論文では二足歩行ロボットに人や物を搭載することを想定し,積載物の転がりと滑りが
% 生じないロボット加速度の範囲を定式化した.
% そして,転がりと滑りが生じない歩幅,ZMP,重心高さの範囲をLIPMを基に定式化した.
% その上で,シミュレーション上にて定式化した歩幅,重心高さの範囲内であれば
% 転がりも滑りも生じないことを確認した.
% 今後の展望として,転がりと滑りが生じない歩幅,ZMP,重心高さを3次元LIPM
% を基に定式化する.そして,シミュレーション,実機を用いて検証を行う.

%%%%%%%%%%%%%%%%%%%%%%%%%%%%%%%%%%%%%%%%%%%%%%%%%%%%%%%%%%%%%%%%%%%%%%%%%%%%%%%%
\section{緒言}
不整地踏破バリカンロボットや階段などの段差が多い建造物内において二足歩行は真価を発揮する.
物を搬送する機械はこれまでにも数多く開発され,それらの多くは
車輪で移動する.しかし,車輪は階段などの段差の移動が難しく運用上に制限がある.
同じく車輪で移動する搬送用機械として電動車椅子\cite{refer5},\cite{refer6}が挙げられる.
電動車椅子も同様に,段差の移動が難しく運用上に制限がある.
これを脚なら,踏み出し動作により段差によらず移動が可能となり,
運用上の制限を格段に少なくできる.
そのため,我々は人や物を搬送できる二足歩行ロボットの開発を行っている\cite{refer9}.\\
 人を搬送する二足歩行ロボットの先行研究として,WLシリーズ\cite{refer1},
搭乗型脚車輪ロボット\cite{refer2},HUBO\cite{refer3},i-foot\cite{refer7}が挙げられる.
これらは人を載せた状態での移動を実現している.
また,物を搬送する二足歩行ロボットとしてAgility
Robotics製のdigit\cite{refer4}が注目されている.
ここで,物や人を搬送する際に着目すべき点として「積載物を倒さず運ぶ」
ことが挙げられる.倒さず運ぶためには積載物が転がらず,滑り出さない
ことが重要である.これは人を載せた際の乗り心地にも影響する.
二足歩行は前後・左右方向に揺れるため積載物を転がさず滑らせないで
運ぶことは難しい.
しかし,先行研究において積載物の転がりと滑りを抑制する歩行方法は
検討されていない.\\
 本論文では二足歩行ロボットに物や人を搭載し,
搬送する場合を想定して,積載物が転がらず滑らない条件を定式化する.
加えて,積載物が転がらず滑らない歩行条件を定式化する.そして,
検討した歩行方法をシミュレータにて
\figref{robot_load.eps}に示すモデルを用いて検証する.
\begin{figure}[t]
    \begin{center}
    \includegraphics[width=70mm]{./fig/robot_load.eps}
    % \vspace*{-0.5cm}%鉛直方向のFig.の位置
    \caption{A load-carrying bipedal robot}
    % \vspace*{-1cm}
    \figlabel{robot_load.eps}
  \end{center}
\end{figure}
% 脚なら段差によらずに移動が可能となり
% 物を搬送するロボットはこれまでにも数多く研究され,それらの多くは
% 車輪で移動する.(参考文献6個)
% 物を搬送するロボットの移動方法として車輪が挙げられる.
% 車輪で移動する

% しかし,先行研究において積載物の転がりと滑りの抑制は考慮されておらず,
% 手やベルト等で固定することが前提となっている.\\

% 物の搬送において物を落とさず様々な場所へ運ぶことは重要である.

% そのため,物を搬送するロボットとして二足歩行ロボットが注目されている(digit).

% そのため,・階段昇降が可能な電動車いすの欠点を列挙する
% 森林地帯などの狭く溝や高低差が多い不整地や階段などの段差が多い建造物内に
% おいて二足歩行は真価を発揮する.
% 物を搬送するロボットは主に車輪によって移動する.
% 車輪だと階段などの段差を登れず,移動できる範囲が狭い.
% 同様の状況下で人を搬送する乗り物として,電動車いすが挙げられる.
% 電動車いすも車輪での移動となるため,移動できる範囲が狭い.
% 脚は踏破性が高く,階段などの段差も登ることができる.
% したがって,我々は以上の問題を解決するために人や物を搬送できる
% 二足歩行ロボットの開発を行っている.\\
%  搭乗型二脚モビリティの先行研究としてはWLシリーズ\cite{refer1},
% 搭乗型脚車輪ロボット\cite{refer2},HUBO\cite{refer3}が挙げられ,
% これらは人を載せた状態での移動を実現している.
% 物や人を搬送する際に着目すべき点として「物や人を落とさず運ぶ」
% ことが挙げられる.落とさず運ぶためには積載物が転がらず,滑り出さない
% ことが前提として挙げられる.歩行は前後,左右方向に揺れるため
% 積載物を転がらず,滑り出さないで運ぶことは難しい.
% 先行研究において,積載物の転がりと滑りの抑制は考慮されていない.
% したがって,本論文では二足歩行ロボットに物や人を搭載し,
% 搬送する場合を想定して,積載物が転がらず滑らない条件を定式化する.
% そして,積載物が転がらず滑らない歩行方法を検討する.そして,
% 検討した歩行方法をシミュレータにて検証する.

% ・搬送用ロボットと電動車いすに着目している
% ・搬送用ロボットの先行研究を列挙
% ・問題点を列挙
% ・問題点を解決する(なぜ二足歩行なのか)
% ・搭乗型ロボットの先行研究を列挙
% ・問題点を列挙
% ・問題点を解決する(なぜ転がりと滑りを防止したいのか)
% ・線形倒立振子モードを使う理由も
% ・本論文では二足歩行ロボットに物や人を搭載し,搬送する場合を想定して,
% 積載物が転がらず滑らない条件を定式化する.そして,積載物が転がらず滑らない
% 歩行方法を線形倒立振子モード(LIPM)(梶田先生論文)\cite{refer1}にて検討する.そして,
% 検討した歩行方法をシミュレータにて検証する.


\section{積載物が転がらず滑らない条件の定式化}
\subsection{積載物とロボットのモデル化}
\label {2.1}
\begin{figure}[t]
    \begin{center}
    \includegraphics[width=70mm]{./fig/robot_load_model_2.eps}
    % \vspace*{-0.5cm}%鉛直方向のFig.の位置
    \caption{Rigid model of a robot body and load}
    % \vspace*{-1cm}
    \figlabel{robot_load_model.eps}
  \end{center}
  \end{figure}
  
ロボットに搭載した物が転がり始めず,滑り始めない条件を定式化するために
積載物とロボットを\figref{robot_load_model.eps}に示す剛体にモデル化した.
積載物モデルの底面形状は長方形とし,質量を${m}_{\mathrm{l}}$
,重心位置を$\bm{c}_{\mathrm{l}}$とする.
一方,ロボットモデルは積載物との接触部分のみを考慮するために,
ボディのみを長方形の剛体としてモデル化し,重心位置を
$\bm{c}_{\mathrm{r}}$とする.
また,ロボットボディの並進運動による影響のみに着目するために,
ロボットボディの慣性モーメントは無限大と
みなし,回転運動は生じないものとする.
積載物とロボットボディ間に作用する
摩擦力の合力を$\bm{f}$,垂直抗力の合力を$\bm{N}$で表し,それら反力の
作用点(圧力中心)を$\bm{p}$とする.$\bm{g}$は重力
加速度である.\\%[m/${\mathrm{s}^2}$]
 ここで,積載物が転がり始めるのは摩擦力$\bm{f}$よる重心周りのモーメント
と垂直抗力$\bm{N}$による重心周りのモーメントの和が$\bm{0}$より大きくなった
ときである.
したがって,積載物が転がり始めない摩擦力の範囲は\equref{1.1}で表される.
\begin{eqnarray}
  \equlabel{1.1}
  \bm{f} &\leq&-{[\bm{s}_{\mathrm{l}}]}_{\mathrm{x}}^{-1}{[\bm{s}_{\mathrm{l}}]}_{\mathrm{x}}\bm{N}\\
  \bm{s}_{\mathrm{l}} &=&\bm{p}-\bm{c}_{\mathrm{l}}\\
  \bm{s}_{\mathrm{l}} &=&\left[
    \begin{array}{ccc}
      {s}_{\mathrm{lx}} & {s}_{\mathrm{ly}} & {s}_{\mathrm{lz}}
    \end{array}
    \right]^{\mathrm{T}}
\end{eqnarray}
ここで,${[ ]}_{\mathrm{x}}$は歪対称行列を表す演算子,
$\bm{s}_{\mathrm{l}}$は積載物の重心から見た時の圧力中心の位置である.

積載物が転がり始める瞬間において,圧力中心$\bm{p}$は積載物底面の辺と端に
移動し,辺と端を回転中心として転がり始める.そのため,転がり始める瞬間
の${s}_{\mathrm{lx}}$,${s}_{\mathrm{ly}}$は積載物の重心投影点と辺までの長さに,
${s}_{\mathrm{lz}}$は重心の底面からの高さに等しくなる.\\

\subsection{積載物の転がりと滑りの条件分の定式化}
転がりが先に起こる場合と滑りが先に起こる場合の条件分を定式化する.
\equref{1.1}で表される積載物が転がり始める
摩擦力の絶対値よりも最大静止摩擦力の絶対値が大きい場合は転がりが
先に起こり,小さい場合は滑りが先に起こる.
したがって,\equref{1.1}より転がりが先に起こる場合と,
滑りが先に起こる場合の条件分は以下で表される.\\
\begin{eqnarray}
  \equlabel{1.2}
  |-{[\bm{s}_{\mathrm{l}}]}_{\mathrm{x}}^{-1}{[\bm{s}_{\mathrm{l}}]}_{\mathrm{x}}\bm{e}_{\mathrm{z}}|\leq{\mu}
\end{eqnarray}
ここで,$\mu$は積載物とロボットボディ間の最大静止摩擦係数,
$\bm{e}_{\mathrm{z}}$はz成分のみ1となる単位ベクトルである.
右辺が$\mu$以下であれば転がりの方が先に起こり,滑りは生じない.
$\mu$より大きければ滑りの方が先に起こり,転がりは生じない.\\
 ここで,$x$方向のみの場合を考えると
\equref{1.2}は\equref{1.3}で表される.\\
\begin{eqnarray}
  \equlabel{1.3}
  \frac{{s}_{\mathrm{lx}}}{{s}_{\mathrm{lz}}}\leq{\mu}
\end{eqnarray}
\equref{1.3}より積載物の転がりと滑りのどちらが先に起こるかは,
積載物をロボットに乗せる前に判断できる.
% 転がりが先に起こる場合は,\equref{1.1}で表される積載物が転がり始める
% 摩擦力の絶対値よりも最大静止摩擦力の絶対値が大きい場合である.
% 一方で滑りが先に起こる場合は,
% 転がり始める摩擦力の絶対値よりも最大静止摩擦力の絶対値が小さい場合である.

\subsection{積載物が転がり始めない条件の定式化}
\label {2.3}
積載物の転がり始めない条件について考える.
積載物とロボットボディ間に生じる摩擦力は主に積載物に作用する重力とロボットボディ
の重心加速度が起因で生じている.また,転がりが生じる瞬間においては積載物とロボットボディ
の間に滑りは生じていないため,${\ddot{\bm{c}}_{\mathrm{l}}}={\ddot{\bm{c}}_{\mathrm{r}}}$が成立する.
したがって,転がりが生じる瞬間におけるロボットボディの加速度の範囲は\equref{1.1}より
\equref{1.4}で表される.\\
\begin{eqnarray}
  \equlabel{1.4}
  {\ddot{\bm{c}}_{\mathrm{rH}}} &\leq& -{[\bm{s}_{\mathrm{l}}]}_{\mathrm{x}}^{-1}{[\bm{s}_{\mathrm{l}}]}_{\mathrm{x}}({\ddot{\bm{c}}_{\mathrm{rV}}}+\bm{g})\\
  \bm{c}_{\mathrm{r}} &=&\left[
    \begin{array}{ccc}
      {c}_{\mathrm{rx}} & {c}_{\mathrm{ry}} & {c}_{\mathrm{rz}}
    \end{array}
    \right]^{\mathrm{T}}\\
  \bm{c}_{\mathrm{rH}} &=&\left[
    \begin{array}{ccc}
      {c}_{\mathrm{rx}} & {c}_{\mathrm{ry}} & 0
    \end{array}
    \right]^{\mathrm{T}}\\
  \bm{c}_{\mathrm{rV}} &=&\left[
    \begin{array}{ccc}
      0 & 0 & {c}_{\mathrm{rz}}
    \end{array}
    \right]^{\mathrm{T}}\\
  \bm{g} &=&\left[
    \begin{array}{ccc}
      0 & 0 & g
    \end{array}
    \right]^{\mathrm{T}}
\end{eqnarray}
% \begin{bmatrix}
%   1 & 0 & 0 \\
%   0 & 1 & 0\\
%   0 & 0 & 0  
% \end{bmatrix}
% \bm{c}_{\mathrm{r}}  \\
% \bm{c}_{\mathrm{rV}} &=&
% \begin{bmatrix}
%   0 & 0 & 0 \\
%   0 & 0 & 0\\
%   0 & 0 & 1  
% \end{bmatrix}
% \bm{c}_{\mathrm{r}} 
 ここで,$x$方向のみの場合を考えると
\equref{1.4}は\equref{1.5}で表される.\\
\begin{eqnarray}
  \equlabel{1.5}
  {\ddot{c}_{\mathrm{rx}}} &\leq& \frac{{s}_{\mathrm{lx}}}{{s}_{\mathrm{lz}}}({\ddot{c}_{\mathrm{rz}}}+{g})
\end{eqnarray}
したがって,\equref{1.5}の範囲内になるようにロボットボディの加速度を制御すれば,
積載物の転がりを抑制できる.

\subsection{積載物が滑り始めない条件の定式化}
\label {2.4}
積載物の滑り始めない条件について考える.
転がりと同様に,滑りが生じる瞬間においては
${\ddot{\bm{c}}_{\mathrm{l}}}={\ddot{\bm{c}}_{\mathrm{r}}}$が成立する.
したがって,滑りが生じる瞬間におけるロボットボディの加速度は\equref{1.6}で表される.\\
\begin{eqnarray}
  \equlabel{1.6}
  |{\ddot{\bm{c}}_{\mathrm{rH}}}| &\leq& \mu({\ddot{c}_{\mathrm{rz}}}+g)
\end{eqnarray}
 ここで,$x$方向のみの場合を考えると
\equref{1.6}は\equref{1.7}で表される.\\
\begin{eqnarray}
  \equlabel{1.7}
  {\ddot{c}_{\mathrm{rx}}} &\leq& \mu({\ddot{c}_{\mathrm{rz}}}+{g})
\end{eqnarray}
したがって,\equref{1.7}の範囲内になるようにロボットボディの加速度を制御すれば,
積載物の滑りを抑制できる.
%  したがって,転がりと同様に積載物が転がり始めないためには\equref{1.7}を
% 満たすようにロボットボディの加速度を制御すればいい.

% 転がりよりも滑りよりが先に起こる場合について考える.\\
%  滑りが生じる瞬間における積載物の加速度は\equref{1.1},〇〇より
% \equref{1.6}で表される.\\
%  このとき,積載物とロボットボディの間に滑りは生じているいないため,〇〇を満たす.
% したがって,転がりが生じる瞬間におけるロボットの加速度は\equref{1.6}で表される.\\
%  ここで,わかりやすくするためにロボットの加速度\equref{1.6}を二次元の場合で考える.
% このとき\equref{1.6}は\equref{1.7}で表される.\\
%  したがって,積載物が転がり始めないためには\equref{1.7}を満たすように
% ロボットのボディの加速度を制御すればいい.


\section{LIPMにおける転がりと滑りの抑制方法の検討}
\label {3}
\begin{figure}[t]
  \begin{center}
  \includegraphics[width=70mm]{./fig/LIPM_2.eps}
  % \vspace*{-0.5cm}%鉛直方向のFig.の位置
  \caption{Linear inverted pendulum mode}
  % \vspace*{-1cm}
  \figlabel{LIPM.eps}
\end{center}
\end{figure}

積載物が転がり始めず,滑り始めないロボットの歩行方法を
検討するために,ロボットを\figref{LIPM.eps}に示す線形倒立振子モード
(LIPM)\cite{refer8}として考える.${p}_{\mathrm{sup}}$は
前後方向における足の接地位置,${c}_{\mathrm{rx}}$は前後方向
重心位置,${c}_{\mathrm{rz}}$は重心高さである.
また,\ref{2.1}節にてモデル化したロボットボディの重心位置とLIPMにおける
ロボットの重心位置は足の慣性を無視し,等しいと考える.\\
 LIPMではロボットの重心高さを一定に保つため,
物を上下に揺らさずに搬送できる.加えて,\equref{1.5}と\equref{1.7}に
含まれる上下方向重心加速度$\ddot{c}_{\mathrm{rz}}$は0 m/${\mathrm{s}^2}$となり,
式が簡単になる.\\
 支持脚切替時の前後方向重心位置を${c}_{\mathrm{rz}}(0)$,
前後方向歩幅を${w}_{\mathrm{x}}=2({c}_{\mathrm{rz}}(0)-{p}_{\mathrm{sup}})$
とすると,LIPMにおけるロボットの重心加速度は支持脚切替時に最大値
${\ddot{c}_{\mathrm{rx}}}(0)=\frac{{w}_{\mathrm{x}}}{2{c}_{\mathrm{rz}}}g$を取る.

\subsection{積載物が転がり始めない歩行方法の検討}
\label {3.1}
まずは積載物が転がり始めない歩行方法をLIPMにおける重心加速度から検討する.
LIPMにおけるロボットの重心加速度の最大値より,歩幅${w}_{\mathrm{x}}$と重心高さ
${c}_{\mathrm{rz}}$によってロボットの重心加速度の最大値が調節できることがわかる.
したがって,\equref{1.5}とロボットの重心加速度の最大値より
積載物が転がり始めない歩幅と重心高さの範囲はそれぞれ
\equref{1.8},\equref{1.9}で表される.\\
\begin{eqnarray}
  \equlabel{1.8}
  {w}_{\mathrm{x}} &\leq& 2\frac{{s}_{\mathrm{lx}}}{{s}_{\mathrm{lz}}}{c}_{\mathrm{rx}}\\
  \equlabel{1.9}
  {c}_{\mathrm{rz}} &\geq& \frac{{s}_{\mathrm{lz}}}{2{s}_{\mathrm{lx}}}{w}_{\mathrm{x}}
\end{eqnarray}
以上から,歩幅,重心高さを調節することで積載物の転がりは抑制できる.
%  また,初期重心位置と接地位置の差分は歩幅に相当するため,\equref{1.8}は
% 歩幅を用いて\equref{1.9}で表すこともできる.\\
%  また,接地位置は足に面積を持つ場合や両足接地時はZMPに相当する.
\subsection{積載物が滑り始めない歩行方法の検討}
\label {3.2}
積載物が滑り始めない歩行方法をLIPMにおける重心加速度から検討する.
転がりと同様に,\equref{1.7}とロボットの重心加速度の最大値より
積載物が滑り始めない歩幅と重心高さの範囲はそれぞれ
\equref{2.0},\equref{2.1}で表される.\\
\begin{eqnarray}
  \equlabel{2.0}
  {w}_{\mathrm{x}} &\leq& 2\mu{c}_{\mathrm{rx}}\\
  \equlabel{2.1}
  {c}_{\mathrm{rz}} &\geq& \frac{{w}_{\mathrm{x}}}{2\mu}
\end{eqnarray}
以上から,転がりと同様に歩幅,重心高さを調節することで積載物の滑りは抑制できる.
%  また,初期重心位置と接地位置の差分は歩幅に相当するため,\equref{2.0}は
% 歩幅を用いて\equref{2.2}で表すこともできる.\\
%  また,接地位置は足に面積を持つ場合や両足接地時はZMPに相当する.
% 以上から,転がりと同様にZMP,歩幅,重心高さを調節することで積載物の滑りは抑制できる.
 
\section{ロボット歩行時における転がりと滑りの抑制方法の検証}
\subsection{シミュレーション条件}
\begin{figure}[t]
    \begin{center}
    \includegraphics[width=60mm]{./fig/robot_3.eps}
    % \vspace*{-0.5cm}%鉛直方向のFig.の位置
    \caption{Model of bipedal robot}
    % \vspace*{-1cm}
    \figlabel{robot.eps}
  \end{center}
\end{figure}

\begin{figure}[t]
    \begin{minipage}[b]{0.45\linewidth}
      \centering
      \includegraphics[width=1.0\linewidth]{./fig/loadA.eps}%
      % \hspace{-10truemm}%水平方向のFig.の位置
      \vspace*{-0.1cm}%鉛直方向のFig.の位置
      \caption{Model of load A}
      \figlabel{loadA.eps}
    \end{minipage}
    \begin{minipage}[b]{0.45\linewidth}
      \centering
      \includegraphics[width=1.0\linewidth]{./fig/loadB.eps}%
    %  \hspace{10truemm}%水平方向のFig.の位置 
      \caption{Model of load B}
      \figlabel{loadB.eps}
    \end{minipage}
    %\caption{subcaptionを用いて図を並べる}
  \end{figure}

  前章で求めた歩幅と重心高さによって積載物が
  転がり始めず,滑り始めないで歩行が可能かを確認するために
シミュレーション上にて検証を行った.使用したシミュレータは
Choreonoidである.使用したロボットのモデルを\figref{robot.eps}に示す.
ロボットモデルは股ヨー・ロール・ピッチ軸,膝ピッチ軸,足首ロール・ピッチ軸の
片足6自由度,全12自由度で構成される.
% ロボットモデルの大きさは人を載せられる程度に設定した.
ロボットボディの慣性モーメントはLIPMにできるだけ近づけるために
$10^4$ kg${\mathrm{m}^2}$に設定した.
積載物のモデルは2つ用意し,\equref{1.3}を用いてパラメータを設定した.
それぞれ,\figref{loadA.eps}に示す転がりが先に起こる場合のモデルをA,\figref{loadB.eps}に示す
滑りが先に起こる場合ののモデルをBとした.
ロボットモデルと2つの積載物モデルの詳細を\tabref{table1}に示す.\\
 歩幅の検証の際には,重心高さを0.7 mに設定した.よって,
\tabref{table1}と\equref{1.8},\equref{2.0}より
転がり始めない歩幅と滑り始めない歩幅は同じく0.35 m以下になった.\\
 重心高さの検証の際には,歩幅を0.3 mに設定した.よって,
\tabref{table1}と\equref{1.9},\equref{2.1}より
転がり始めない重心高さと滑り始めない重心高さは同じく0.68 m以上になった.\\
 以上より検証条件を\tabref{table2}に示すように設定した.
積載物の縦幅は$2{s}_{\mathrm{lz}}$,横幅は$2{s}_{\mathrm{lx}}$に相当する.
\tabref{table2}を満たすように5 s間ロボットを積載物を載せた状態で歩行させる.
% モデル誤差や横方向加速度の影響等を考慮した上で,\tabref{table2}に示すように設定した.
\subsection{ロボット歩行時の積載物の転がり抑制の検証}
\subsubsection{歩幅を調節した結果}
\label {4.1.1}
歩幅を調節することで積載物の転がりを抑制できるか確認を行った.
歩幅0.3 m,0.4 mで歩行した際の積載物Aのピッチ角を\figref{pitch_0.3_0.4_r.pdf}に示す.\\
 \figref{pitch_0.3_0.4_r.pdf}より,歩幅0.35 m以上の歩幅0.4 mで歩行した際は
積載物のピッチ角の最大値が7.4 degとなり,転がりが生じた.
一方で,歩幅0.35 m以下の歩幅0.3 mで歩行した際は積載物のピッチ角の最大値が0.16 degとなり,
歩幅0.4 mと比べてピッチ角が97.8 \%抑えられた.
ピッチ角0.16degはボディへのめり込みによるものであり,
転がりによるものではないと考えられる.
したがって,\equref{1.8}により求めた歩幅の範囲内において転がりを防止できた.\\
%  また,\figref{xdisp_0.3_0.4_r.pdf}より歩幅0.3 mの時は相対変位の最大値が
% 約5.9 mmとなっており,滑りも抑制されている.
% 一方で歩幅0.4 mの時は相対変位の最大値が約30.9 mmとなっている.
% これは,ピッチ角の転がりが生じたことで積載物の重心位置が変化した結果である.
%  歩幅0.3 m,0.4 mで歩行した際の積載物Aのピッチ角とロボットボディと
% 積載物間の相対変位をそれぞれ\figref{pitch_0.3_0.4_r.pdf}と
% \figref{xdisp_0.3_0.4_r.pdf}に示す.\\
\subsubsection{重心高さを調節した結果}
\label {4.1.2}
重心高さを調節することで積載物の転がりを抑制できるか確認を行った.
重心高さ0.7 m,0.55 mで歩行した際の積載物Aのピッチ角を
\figref{pitch_0.7_0.55_r.pdf}に示す.\\
 \figref{pitch_0.7_0.55_r.pdf}より,
重心高さ0.68 m以下の歩幅0.55 mで歩行した際は
積載物のピッチ角の最大値が90 degとなり,転がりが生じた後に落下した.
一方で,重心高さ0.68 m以上の重心高さ0.7 mで歩行した際は積載物のピッチ角の
最大値が0.16 degとなり,歩幅0.4 mと比べてピッチ角が99.8 \%抑えられた.
\ref{4.1.1}節と同様に,ピッチ角0.16degはボディへのめり込みによるものであり,
転がりによるものではないと考えられる.
したがって,\equref{1.9}により求めた重心高さの範囲内において転がりを防止できた.\\
%  また,\figref{xdisp_0.7_0.55_r.pdf}より歩幅0.3 mの時は相対変位の最大値が約5.9 mmとなって
% おり,滑りも抑制されている.一方で歩幅0.4 mの時は相対変位の最大値が約804 mmとなっている.
% これは,積載物が落下したことによって生じたものである.
%  重心高さ0.7 m,0.55 mで歩行した際の積載物Aのピッチ角とロボットボディと
% 積載物間の相対変位をそれぞれ\figref{pitch_0.7_0.55_r.pdf}と\figref{xdisp_0.7_0.55_r.pdf}
% に示す.\\
\begin{table}[t]
    % \vspace*{0.4cm}%鉛直方向のFig.の位置
    \caption{Parameters of the model}
    \tablabel{table1}
    \begin{center}
        
    \scalebox{0.88}{
    \begin{tabular}{|l | r | r | r|}%1) 2a) 2b) 3) 振幅 振動数 波形 

        \hline
        モデル名&質量&大きさ&最大静止摩擦係数\\ 
        \hline
        ロボット&20 kg&全高 1.2 m&-\\ 
        
        積載物A&0.5 kg&縦・横幅 0.4,0.1 m&0.5\\         
        
        積載物B&0.5 kg&縦・横幅 0.3,0.1 m&0.25\\ 
        \hline
    \end{tabular}
    }
    \end{center}
\end{table}

\begin{table}[t]
    % \vspace*{0.4cm}%鉛直方向のFig.の位置
    \caption{Verification condition}
    \tablabel{table2}
    \begin{center}
        
    \scalebox{0.9}{
    \begin{tabular}{|l | r | r | r | r |}%1) 2a) 2b) 3) 振幅 振動数 波形 

        \hline
        場合分&使用モデル&調整したパラメータ&パラメータ値\\ 
        \hline
        転がり&積載物A&歩幅&0.3 m,0.4 m\\ 
        転がり&積載物A&重心高さ&0.55 m,0.7 m\\
        滑り&積載物B&歩幅&0.3 m,0.4 m\\         
        滑り&積載物B&重心高さ&0.55 m,0.7 m\\ 
        \hline
    \end{tabular}
    }
    \end{center}
\end{table}

\subsection{ロボット歩行時の積載物の滑り抑制の検証}
\subsubsection{歩幅を調節した結果}
\label {4.1.3}
歩幅を調節することで積載物の滑りを抑制できるか確認を行った.
歩幅0.3 m,0.4 mで歩行した際のロボットボディと
積載物B間の相対変位を\figref{xdisp_0.3_0.4_s.pdf}に示す.\\
 \figref{xdisp_0.3_0.4_s.pdf}より,歩幅0.35 m以上の歩幅0.4 mで歩行した際は
ロボットボディと積載物B間の相対変位の最大値が21.2 mmとなった.
一方で,歩幅0.35 m以下の歩幅0.3 mで歩行した際は相対変位の最大値が16.3 mmとなり,
歩幅0.4 mと比べて相対変位が23 \%抑えられた.
したがって,\equref{2.0}により求めた歩幅の範囲内において滑りを軽減できた.
滑りが生じた原因としては,ロボットボディの横方向加速度,ロボットのモデル化誤差,
Choreonoidの滑りのモデルの誤差等が考えられる.\\
%  また,\figref{pitch_0.3_0.4_s.pdf}より歩幅0.3 m,0.4 mの時のピッチ角の最大値が
% それぞれ0.09 deg,0.06 degとなっており,転がりは抑制されている.
%  歩幅0.3 m,0.4 mで歩行した際の積載物Bのピッチ角とロボットボディと
% 積載物間の相対変位をそれぞれ\figref{pitch_0.3_0.4_s.pdf}と\figref{xdisp_0.3_0.4_s.pdf}
% に示す.\\

\subsubsection{重心高さを調節した結果}
重心高さを調節することで積載物の滑りを抑制できるか確認を行った.
重心高さ0.7 m,0.55 mで歩行した際のロボットボディと
積載物B間の相対変位を\figref{xdisp_0.7_0.55_s.pdf}に示す.\\
 \figref{xdisp_0.7_0.55_s.pdf}より,重心高さ0.68 m以下の重心高さ0.55 mで
歩行した際はロボットボディと積載物B間の相対変位の最大値が31.4 mmとなった.
一方で,重心高さ0.68 m以下の重心高さ0.7 mで歩行した際は相対変位の最大値が
16.3 mmとなり,重心高さ0.55 mと比べて相対変位が48 \%抑えられた.
したがって,\equref{2.1}により求めた重心高さの範囲内において滑りを軽減できた.
滑りが生じた原因としては,\ref{4.1.3}節と同様であると考えられる.\\
%  また,\figref{pitch_0.7_0.55_s.pdf}より重心高さ0.7 m,0.55 mの時のピッチ角の最大値が
% それぞれ0.09 deg,0.012 degとなっており,転がりも抑制されている.
%  重心高さ0.7 m,0.55 mで歩行した際の積載物のピッチ角とロボットボディと
% 積載物間の相対変位をそれぞれ\figref{pitch_0.7_0.55_s.pdf}と\figref{xdisp_0.7_0.55_s.pdf}
% に示す.\\

\begin{figure}[tb]
    \begin{center}
    \includegraphics[width=90mm]{./fig/pitch_0.3_0.4_r.pdf}
    % \vspace*{-0.5cm}%鉛直方向のFig.の位置
    \caption{Pitch angle of load when the step length is adjusted}
    % \vspace*{-1cm}
    \figlabel{pitch_0.3_0.4_r.pdf}
\end{center}
\end{figure}
% \begin{figure}[tb]
%     \begin{center}
%     \includegraphics[width=90mm]{./fig/xdisp_0.3_0.4_r.pdf}
%     % \vspace*{-0.5cm}%鉛直方向のFig.の位置
%     \caption{歩幅を調節した時の積載物Aとロボットボディの相対変位}
%     % \vspace*{-1cm}
%     \figlabel{xdisp_0.3_0.4_r.pdf}
% \end{center}
% \end{figure}
\begin{figure}[tb]
    \begin{center}
    \includegraphics[width=90mm]{./fig/pitch_0.7_0.55_r.pdf}
    % \vspace*{-0.5cm}%鉛直方向のFig.の位置
    \caption{Pitch angle of load when the CoG height is adjusted}
    % \vspace*{-1cm}
    \figlabel{pitch_0.7_0.55_r.pdf}
\end{center}
\end{figure}
% \begin{figure}[tb]
%     \begin{center}
%     \includegraphics[width=90mm]{./fig/xdisp_0.7_0.55_r.pdf}
%     % \vspace*{-0.5cm}%鉛直方向のFig.の位置
%     \caption{重心高さを調節した時の積載物Aとロボットボディの相対変位}
%     % \vspace*{-1cm}
%     \figlabel{xdisp_0.7_0.55_r.pdf}
% \end{center}
% \end{figure}
% \begin{figure}[tb]
%     \begin{center}
%     \includegraphics[width=90mm]{./fig/pitch_0.3_0.4_s.pdf}
%     % \vspace*{-0.5cm}%鉛直方向のFig.の位置
%     \caption{歩幅を調節した時の積載物Bのピッチ角}
%     % \vspace*{-1cm}
%     \figlabel{pitch_0.3_0.4_s.pdf}
% \end{center}
% \end{figure}
\begin{figure}[t]
    \begin{center}
    \includegraphics[width=90mm]{./fig/xdisp_0.3_0.4_s.pdf}
    % \vspace*{-0.5cm}%鉛直方向のFig.の位置
    \caption{Displacement between the robot and the load when the step length is adjusted}
    % \vspace*{-1cm}
    \figlabel{xdisp_0.3_0.4_s.pdf}
\end{center}
\end{figure}
% \begin{figure}[tb]
%     \begin{center}
%     \includegraphics[width=90mm]{./fig/pitch_0.7_0.55_s.pdf}
%     % \vspace*{-0.5cm}%鉛直方向のFig.の位置
%     \caption{重心高さを調節した時の積載物Bのピッチ角}
%     % \vspace*{-1cm}
%     \figlabel{pitch_0.7_0.55_s.pdf}
% \end{center}
% \end{figure}
\begin{figure}[tb]
    \begin{center}
    \includegraphics[width=90mm]{./fig/xdisp_0.7_0.55_s.pdf}
    % \vspace*{-0.5cm}%鉛直方向のFig.の位置
    \caption{Displacement between the robot and the load when the CoG height is adjusted}
    % \vspace*{-1cm}
    \figlabel{xdisp_0.7_0.55_s.pdf}
\end{center}
\end{figure}



% \input{chap_5}
% \input{chap_6}

% 1.緒言

% 転がり・滑り始めない条件の定式化
% 積載物とロボットのモデル化
% ・一般的なモデル
% ・長方形・二次元の場合
% 転がりと滑りの条件わけ
% ・一般的な場合
% ・長方形・二次元の場合
% 転がりの方が先に起こる場合
% ・一般的な場合
% ・長方形・二次元の場合
% 滑りの方が先に起こる場合
% ・一般的な場合
% ・長方形・二次元の場合

% LIPMでの転がり・滑りの回避方法の検討
% 転がりの方が先に起こる場合
% 滑りの方が先に起こる場合
% メモ:長方形・二次元で考える

% シミュレータによる検証
% 転がりの方が先に起こる場合
% 滑りの方が先に起こる場合

% 結言

%%%%%%%%%%%%%%%%%%%%%%%%%%%%%%%%%%%%%%%%%%%%%%%%%%%%%%%%%%%%%%%%%%%%%%%%%%%%%%%%


\section{CONCLUSIONS}
本論文では二足歩行ロボットに物や人を搭載して搬送する場合を想定し,
積載物が転がらず滑らないロボットボディ加速度の範囲を定式化した.
そして,積載物が転がらず滑らない歩幅,重心高さをLIPMを基に定式化した.
そして,シミュレーション上にて定式化した歩幅,重心高さの範囲内であれば
歩行中の積載物の転がりを防止し,滑りを軽減できることを確認した.\\
 今後の展望としては,本論文の検証内容を我々が開発した
小型の実機にて検証実験を行う.
加えて,積載物の転がらず滑らない条件を一般化し,それを満たす歩行方法を3次元LIPMを基に定式化する.
そして,シミュレーションと実機を用いて検証を行う.
最終的には,人が乗れるサイズの実機を開発し,積載物を落とさない搬送を実現する.
% 本論文では二足歩行ロボットに人や物を搭載することを想定し,積載物の転がりと滑りが
% 生じないロボット加速度の範囲を定式化した.
% そして,転がりと滑りが生じない歩幅,ZMP,重心高さの範囲をLIPMを基に定式化した.
% その上で,シミュレーション上にて定式化した歩幅,重心高さの範囲内であれば
% 転がりも滑りも生じないことを確認した.
% 今後の展望として,転がりと滑りが生じない歩幅,ZMP,重心高さを3次元LIPM
% を基に定式化する.そして,シミュレーション,実機を用いて検証を行う.
% A conclusion section is not required. Although a conclusion may review the main points of the paper, do not replicate the abstract as the conclusion. A conclusion might elaborate on the importance of the work or suggest applications and extensions. 


% \balance
%  \addtolength{\textheight}{-2cm}   % This command serves to balance the column lengths
          %12                        % on the last page of the document manually. It shortens
                                  % the textheight of the last page by a suitable amount.
                                  % This command does not take effect until the next page
                                  % so it should come on the page before the last. Make
                                  % sure that you do not shorten the textheight too much.

%%%%%%%%%%%%%%%%%%%%%%%%%%%%%%%%%%%%%%%%%%%%%%%%%%%%%%%%%%%%%%%%%%%%%%%%%%%%%%%%

\nocite{*}
\bibliography{reference}
\bibliographystyle{junsrt}

\end{document}
